\chapter{附录B:SMO算法}

\emph{代码55}

\begin{lstlisting}[language=python]
import numpy as np 
from random import randrange 

# Written from the pseudo-code in: 
# http://luthuli.cs.uiuc.edu/~daf/courses/optimization/Papers/smoTR.pdf 
class SmoAlgorithm: 
    def __init__(self, X, y, C, tol, kernel, use_linear_optim): 
        self.X = X 
        self.y = y 
        self.m, self.n = np.shape(self.X) 
        self.alphas = np.zeros(self.m) 
        
        self.kernel = kernel 
        self.C = C 
        self.tol = tol 
        
        self.errors = np.zeros(self.m) 
        self.eps = 1e-3 # epsilon 
        
        self.b = 0 

        self.w = np.zeros(self.n) 
        self.use_linear_optim = use_linear_optim 
        
    # Compute the SVM output for example i 
    # Note that Platt uses the convention w.x-b=0 
    # while we have been using w.x+b in the book. 
    def output(self, i): 
        if self.use_linear_optim: 
            # Equation 1 
            return float(np.dot(self.w.T, self.X[i])) - self.b 
            else: 
                # Equation 10 
                return np.sum([self.alphas[j] * self.y[j] * self.kernel(self.X[j], self.X[i]) for j in range(self.m)]) - self.b 
                
    # Try to solve the problem analytically. 
    def take_step(self, i1, i2):
        if i1 == i2: 
            return False 
            
        a1 = self.alphas[i1] 
        y1 = self.y[i1] 
        X1 = self.X[i1] 
        E1 = self.get_error(i1) 
        
        s = y1 * self.y2 
        
        # Compute the bounds of the new alpha2. 
        if y1 != self.y2: 
            # Equation 13 
            L = max(0, self.a2 - a1) 
            H = min(self.C, self.C + self.a2 - a1) 
        else: 
            # Equation 14 
            L = max(0, self.a2 + a1 - self.C) 
            H = min(self.C, self.a2 + a1) 
            
        if L == H: 
            return False 
            
        k11 = self.kernel(X1, X1) 
        k12 = self.kernel(X1, self.X[i2]) 
        k22 = self.kernel(self.X[i2], self.X[i2])
        
        # Compute the second derivative of the # objective function along the diagonal. 
        # Equation 15 
        eta = k11 + k22 - 2 * k12 
        
        if eta > 0: 
            # Equation 16 
            a2_new = self.a2 + self.y2 * (E1 - self.E2) / eta 
            
            # Clip the new alpha so that is stays at the end of the line. 
            # Equation 17 
            if a2_new < L: 
                a2_new = L 
            elif a2_new > H: 
                a2_new = H 
        else: 
            # Under unusual cicumstances, eta will not be positive.
            # Equation 19 
            f1 = y1 * (E1 + self.b) - a1 * k11 - s * self.a2 * k12 
            f2 = self.y2 * (self.E2 + self.b) - s * a1 * k12 \ - self.a2 * k22 
            L1 = a1 + s(self.a2 - L)
            H1 = a1 + s * (self.a2 - H) 
            Lobj = L1 * f1 + L * f2 + 0.5 * (L1 ** 2) * k11 \ + 0.5 * (L ** 2) * k22 + s * L * L1 * k12 
            Hobj = H1 * f1 + H * f2 + 0.5 * (H1 ** 2) * k11 \ + 0.5 * (H ** 2) * k22 + s * H * H1 * k12 
            if Lobj < Hobj - self.eps: 
                a2_new = L 
            elif Lobj > Hobj + self.eps: 
                a2_new = H 
            else: a2_new = self.a2 
            
        # If alpha2 did not change enough the algorithm 
        # returns without updating the multipliers. 
        if abs(a2_new - self.a2) < self.eps * (a2_new + self.a2 \ + self.eps): 
            return False 
            
        # Equation 18 
        a1_new = a1 + s * (self.a2 - a2_new) 
        
        new_b = self.compute_b(E1, a1, a1_new, a2_new, k11, k12, k22, y1) 
        
        delta_b = new_b - self.b 
        
        self.b = new_b 
        
        # Equation 22 
        if self.use_linear_optim: 
            self.w = self.w + y1*(a1_new - a1)*X1 \ + self.y2*(a2_new - self.a2) * self.X2 
        
        # Update the error cache using the new Lagrange multipliers. 
        delta1 = y1 * (a1_new - a1) 
        delta2 = self.y2 * (a2_new - self.a2) 
        
        # Update the error cache. 
        for i in range(self.m): 
            if 0 < self.alphas[i] < self.C: 
                self.errors[i] += delta1 * self.kernel(X1, self.X[i]) + delta2 * self.kernel(self.X2,self.X[i]) - delta_b 
            
        self.errors[i1] = 0 
        self.errors[i2] = 0 

        self.alphas[i1] = a1_new 
        self.alphas[i2] = a2_new

        return True 
    
    def compute_b(self, E1, a1, a1_new, a2_new, k11, k12, k22, y1): 
        # Equation 20 
        b1 = E1 + y1 * (a1_new - a1) * k11 + \ self.y2 * (a2_new - self.a2) * k12 + self.b 
        
        # Equation 21 
        b2 = self.E2 + y1 * (a1_new - a1) * k12 + \ self.y2 * (a2_new - self.a2) * k22 + self.b 
        
        if (0 < a1_new) and (self.C > a1_new): 
            new_b = b1 
        elif (0 < a2_new) and (self.C > a2_new): 
            new_b = b2 
        else: 
            new_b = (b1 + b2) / 2.0 
        return new_b 
    
    def get_error(self, i1): 
        if 0 < self.alphas[i1] < self.C: 
            return self.errors[i1] 
        else: 
            return self.output(i1) - self.y[i1] 
            
    def second_heuristic(self, non_bound_indices): 
        i1 = -1 
        if len(non_bound_indices) > 1: 
            max = 0 
            
            for j in non_bound_indices: 
                E1 = self.errors[j] - self.y[j] 
                step = abs(E1 - self.E2) # approximation 
                if step > max: 
                    max = step 
                    i1 = j 
        return i1 
        
    def examine_example(self, i2): 
        self.y2 = self.y[i2] 
        self.a2 = self.alphas[i2] 
        self.X2 = self.X[i2] 
        self.E2 = self.get_error(i2) 
        
        r2 = self.E2 * self.y2 
        
        if not((r2 < -self.tol and self.a2 < self.C) or (r2 > self.tol and self.a2 > 0)):
            # The KKT conditions are met, SMO looks at another example. 
            return 0 
        
        # Second heuristic A: choose the Lagrange multiplier which 
        # maximizes the absolute error. 
        non_bound_idx = list(self.get_non_bound_indexes()) 
        i1 = self.second_heuristic(non_bound_idx) 
        
        if i1 >= 0 and self.take_step(i1, i2): 
            return 1 
            
        # Second heuristic B: Look for examples making positive 
        # progress by looping over all non-zero and non-C alpha, 
        # starting at a random point. 
        if len(non_bound_idx) > 0: 
            rand_i = randrange(len(non_bound_idx)) 
            for i1 in non_bound_idx[rand_i:] + non_bound_idx[:rand_i]: 
                if self.take_step(i1, i2): 
                    return 1 
                    
        # Second heuristic C: Look for examples making positive progress 
        # by looping over all possible examples, starting at a random 
        # point. 
        rand_i = randrange(self.m) 
        all_indices = list(range(self.m)) 
        for i1 in all_indices[rand_i:] + all_indices[:rand_i]: 
            if self.take_step(i1, i2): 
                return 1 
                
        # Extremely degenerate circumstances, SMO skips the first example. 
        return 0 
    
    def error(self, i2): 
        return self.output(i2) - self.y2 
        
    def get_non_bound_indexes(self): 
        return np.where(np.logical_and(self.alphas > 0, self.alphas < self.C))[0] 
        
    # First heuristic: loop over examples where alpha is not 0 and not C 
    # they are the most likely to violate the KKT conditions 
    # (the non-bound subset). 
    def first_heuristic(self): 
        num_changed = 0 
        non_bound_idx = self.get_non_bound_indexes() 
        for i in non_bound_idx: 
            num_changed += self.examine_example(i) 
        return num_changed

    def main_routine(self): 
        num_changed = 0 
        examine_all = True 
        
        while num_changed > 0 or examine_all: 
            num_changed = 0 
            
            if examine_all: 
                for i in range(self.m): 
                    num_changed += self.examine_example(i) 
            else: 
                num_changed += self.first_heuristic() 
            
            if examine_all: 
                examine_all = False 
            elif num_changed == 0: 
                examine_all = True
\end{lstlisting}

代码56演示了如何实例化一个SmoAlgorithm对象、运行算法并打印结果。

\emph{代码56}

\begin{lstlisting}[language=python]
import numpy as np 
from random import seed 
from succinctly.datasets import linearly_separable, get_dataset 
from succinctly.algorithms.smo_algorithm import SmoAlgorithm 

def linear_kernel(x1, x2): 
    return np.dot(x1, x2) 
    
def compute_w(multipliers, X, y): 
    return np.sum(multipliers[i] * y[i] * X[i] for i in range(len(y))) 
    
if __name__ == '__main__': 
    seed(5) # to have reproducible results 
    
    X_data, y_data = get_dataset(linearly_separable.get_training_examples) 
    smo = SmoAlgorithm(X_data, y_data, C=10, tol=0.001, kernel=linear_kernel, use_linear_optim=True) 
    
    smo.main_routine() 
    
    w = compute_w(smo.alphas, X_data, y_data)
    
    print('w = {}'.format(w)) 
    
    # -smo.b because Platt uses the convention w.x-b=0 
    print('b = {}'.format(-smo.b)) 
    
    # w = [0.4443664 1.1105648] 
    # b = -9.66268641132
\end{lstlisting}