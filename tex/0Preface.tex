\chapter*{序}

\section{本书的受众}

这本书的目的是提供支持向量机(svm)的一般概述。您将了解它们是什么,它们可以解决哪些类型的问题,以及如何使用它们。我试图让这本书对许多类别的读者都有用。软件工程师会发现大量的代码示例和简单的算法解释。更深入地理解svm的内部工作方式将使您能够更好地使用可用的实现。

希望第一次看到支持向量机的学生会发现该主题的覆盖面足够大,从而激起他们的好奇心。我还尽可能多地收录参考文献,这样感兴趣的读者就能更深入地阅读。

\section{该如何阅读这本书}

因为每一章都是建立在前一章的基础上,所以按顺序阅读这本书是最好的方法。

\section{参考文献}

在书的末尾你可以找到参考书目。引用一篇论文或一本书时,要用作者的名字加上出版日期。例如,(Bishop, 2006)在参考书目中引用了下面一行:

Bishop, C. M. (2006). Pattern Recognition and Machine Learning. Springer.

\section{代码}

这本书使用的IDE是Pycharm, 版本是Community Edition 2016.2.3,使用的python版本为 WinPython 64-bit 3.5.1.2以及NumPy。你可以在[这里](https://bitbucket.org/syncfusiontech/svm-succinctly)找到本书的所有代码。

