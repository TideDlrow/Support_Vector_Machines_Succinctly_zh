\documentclass{ctexbook}
\usepackage{ctex}
\usepackage{amsmath,amssymb,amsfonts}
\usepackage{amsthm}
\usepackage{mathrsfs}
\usepackage[title]{appendix}%
\usepackage{hyperref}
\hypersetup{
    colorlinks=true,
    linkcolor=blue,
    filecolor=blue,
    urlcolor=blue,
    citecolor=cyan,
}
\usepackage{subfig,graphicx}
\graphicspath{{../pic/}}

\usepackage{caption}
\usepackage[dvipsnames]{xcolor}  % 更全的色系
\usepackage{listings}  % 排代码用的宏包

%%%%%%%%%%%%%%%%%%%%%%%%%%%%%%%%%%%%%%%%
%% listings设置
%%%%%%%%%%%%%%%%%%%%%%%%%%%%%%%%%%%%%%%%
\lstset{
    language = Python,
    backgroundcolor = \color[RGB]{245,245,244},    % 背景色
    basicstyle = \small\ttfamily,           % 基本样式 + 小号字体
    rulesepcolor= \color{gray},             % 代码块边框颜色
    breaklines = true,                  % 代码过长则换行
    numbers = left,                     % 行号在左侧显示
    numberstyle = \small,               % 行号字体
    keywordstyle = \color[RGB]{40,40,255},    % 关键字颜色
    commentstyle =\it\color[RGB]{0,96,96},    % 注释格式和颜色
    stringstyle = \rmfamily\slshape\color[RGB]{128,0,0},    % 字符串格式
    % frame = shadowbox,                  % 用(带影子效果)方框框住代码块
    showspaces = false,                 % 不显示空格
    columns = fixed,                    % 字间距固定
    %escapeinside={<@}{@>}              % 特殊自定分隔符:<@可以自己加颜色@>
    morekeywords = {as},                % 自加新的关键字(必须前后都是空格)
    deletendkeywords = {compile}        % 删除内定关键字;删除错误标记的关键字用deletekeywords删!
}

\begin{document}
    
\title{简要支持向量机}
\author{}

\maketitle

\tableofcontents

% \mainmatter % 开始主要内容

\include{0Preface}
\chapter*{引言}

支持向量机是目前最高效的有监督机器学习算法之一。这意味着当您遇到问题并尝试在其上运行SVM时,您通常会得到非常好的结果,而无需进行很多调整。尽管如此,因为它基于强大的数学背景,它经常被视为一个黑盒子。在本书中,我们将深入讨论SVM背后的主要思想。有几种支持向量机,这就是为什么我经常提到支持向量机。本书的目标是了解它们是如何工作的。

支持向量机是几个人多年工作的结果。第一个SVM算法是Vladimir Vapnik在1963年提出的。他后来与Alexey Chervonenkis就\href{https://en.wikipedia.org/wiki/Vapnik–Chervonenkis_theory}{VC理论}进行了密切的合作,VC理论试图从统计学的角度解释学习过程,他们都对支持向量机做出了巨大的贡献。你可以在\href{http://www.svms.org/history.html}{这里}找到支持向量机的详细历史。

在现实生活中,支持向量机已成功应用于三个主要领域:文本分类、图像识别和生物信息学(Cristianini \& Shawe-Taylor, 2000)。具体的例子包括分类新闻报道、手写数字识别和癌症组织样本。

在第一章中,我们将了解重要的概念:向量、线性可分性和超平面。它们是让您理解支持向量机的基础知识。在第二章中,我们将学习一种称为感知器的简单算法,而不是直接进入这个主题。不要跳过它——尽管它没有讨论支持向量机,本章将给您提供宝贵的见解,解释为什么支持向量机更擅长分类数据。

第三章将逐步构建所谓的支持向量机优化问题。第4章可能是最难的,它将向您展示如何解决这个问题——首先用数学方法,然后用编程方法。在第五章中,我们将发现一种新的支持向量机,称为软间隔支持向量机。我们将看到它是如何对原来的问题进行改进的。

第6章将介绍核,并解释所谓的“核技巧”。有了这个技巧,我们将得到目前最常用的核函数SVM。在第七章中,我们将学习SMO,这是一种专门为快速解决SVM优化问题而创建的算法。在第8章中,我们将看到支持向量机可以用于对多个类别的样本进行分类。

现在让我们开始我们的旅程。
\include{1Prerequisites}
\include{2ThePreceptron}
\chapter{支持向量机优化问题}

\section{支持向量机搜索最优超平面}

感知机有几个优点:它是一个简单的模型,算法很容易实现,我们有一个理论证明它会找到一个超平面来分离数据。但是,它最大的缺点是不能每次都找到相同的超平面。我们为什么要关心这个?因为不是所有分离的超平面都是相等的。如果感知机给你一个超平面它非常接近一个类的所有数据点,你有权利相信它在给出新数据时泛化效果很差。

支持向量机(Support Vector Machines,SVM)没有这个问题。事实上,SVM不是寻找超平面,而是试图得到超平面。我们称它为\textbf{最优超平面},我们说它是最能分离数据的超平面。

\section{怎么比较两个超平面}
因为我们不能根据感觉来选择最优的超平面,我们需要某种评价标准来允许我们比较两个超平面,并说哪一个比其他所有超平面都好。

在本节中,我们将尝试了解如何比较两个超平面。换句话说,我们将寻找一种计算的方法,让我们知道哪个超平面分割数据的效果最好。我们将着眼于看似有效的方法,但随后我们将看到为什么它们不起作用,以及我们如何纠正它们的局限性。让我们只用超平面的方程来比较两个超平面。

\subsection{利用超平面方程}

给定一个样本$(\mathbf{x},y)$和一个超平面,我们想知道这个样本和超平面间有什么关系。

我们已经知道的一个关键元素是如果的值$\mathbf{x}$满足直线方程,那么就表示它在直线上。超平面也类似,对于由向量$\mathbf{w}$和偏移量$b$定义的超平面,如果点$\mathbf{x}$在超平面上,则$\mathbf{w} \cdot \mathbf{x} + b = 0$

但是如果点不在超平面上呢?

让我们看看图\ref{figure22}的例子。在图\ref{figure22}中直线由$\mathbf{w}=(-0.4,-1)$和$b=9$定义,当我们用超平面方程时:
\begin{itemize}
    \item 对于点A(1,3),用向量$\mathbf{a}=(1,3)$,可以得到$\mathbf{w}\cdot\mathbf{a}+b=5.6$
    \item 对于点B(3,5),用向量$\mathbf{b}=(3,5)$,可以得到$\mathbf{w}\cdot\mathbf{b}+b=2.8$
    \item 对于点A(5,7),用向量$\mathbf{c}=(5,7)$,可以得到$\mathbf{w}\cdot\mathbf{c}+b=0$
\end{itemize}

\begin{figure}[ht]
	\centering
	\includegraphics{figure22}
	\caption{A在方程中得到的结果比B大}
	\label{figure22}
\end{figure}

正如你所看到的,当点不在超平面上时我们得到一个不等于0的数。事实上,离超平面越远,得到的数值越大。

另一件需要注意的事情是,方程返回的数字的符号告诉我们这个点相对于直线的位置。利用图\ref{figure23}所示的直线方程,我们得到:
\begin{itemize}
    \item 点A(3,5)得到的是2.8
    \item 点B(5,7)得到的是0
    \item 点C(7,9)得到的是-2.8
\end{itemize}


\begin{figure}[ht]
	\centering
	\includegraphics{figure23}
	\caption{这个方程对C返回负数}
	\label{figure23}
\end{figure}

如果方程返回一个正数,则该点在这条线的下面,如果它是一个负数,则该点在这条线的上面。注意,在视觉上它并不一定在上面或下面,因为如果您有一条像图\ref{figure24}中那样的线,它将在左边或右边,但应用相同的逻辑。超平面方程返回的数字的符号可以让我们判断两点是否在同一侧。事实上,这正是我们在第二章中定义的假设函数的作用。



\begin{figure}[ht]
	\centering
	\includegraphics{figure24}
	\caption{一条线可以以不同的方式分隔空间}
	\label{figure24}
\end{figure}

现在我们有了比较两个超平面的方法。


给定一个训练样本$(\mathbf{x},y)$和一个由向量$\mathbf{w}$和偏移量$b$定义的超平面,我们计算$\beta=\mathbf{w}\cdot\mathbf{x}+b$来知道点到超平面有多远。

给定一个数据集$\mathcal{D}=\left\{(\mathbf{x}_i,y_i)\mid \mathbf{x}_i \in \mathbb{R}^n,y_i \in \{+1,-1\}  \right\}_{i=1}^m$,我们对每个样本计算它的$\beta$值,并记$B$是所有$\beta$中最小的值:
\begin{gather*}
B = \min_{i=1\dots m}\beta_i
\end{gather*}

如果我们需要在两个超平面之间进行选择,我们将选择其中$B$最大的一个。

要清楚的是,这意味着如果我们有$k$个超平面,我们将计算$B_i (i=1\dots k)$并选择其中$B$最大的超平面。

\subsection{负样本存在问题}

不幸的是,利用超平面方程的结果有其局限性。问题是,取最小值对负值的样本(即方程返回负值的样本)是行不通的。

记住,我们总是希望取离超平面最近的点的$\beta$。用正样本来计算$\beta$实际上是没问题 的。如果有两个点的$\beta$分别为$\beta=+5,\beta=+1$,我们会选择较小的那个,也就是$+1$。但是,如果两个值分别为$\beta=-5,\beta=-1$,根据规则会选择$-5$,因为$-5<-1$,实际情况是$\beta=-1$的点会更靠近超平面。


解决这个问题的一种方法是使用的绝对值。

给定一个数据集$\mathcal{D}$,我们对每个样本计算它的$\beta$值,并记$B$是所有$\beta$中绝对值最小的值:
\begin{gather*}
B = \min_{i=1\dots m}|\beta_i|
\end{gather*}

\subsection{超平面是否正确分类数据?}

计算$B$可以让我们选择一个超平面。但是,如果只使用这个值,我们可能会选择一个错误的超平面。考虑图\ref{figure25}中的情况:样本被\textbf{正确分类},使用最后一个公式计算的$B$值为2。


\begin{figure}[ht]
	\centering
	\includegraphics{figure25}
	\caption{超平面正确分类了数据}
	\label{figure25}
\end{figure}

在图\ref{figure26}中,样本分类不正确,但其$B$的值也是2。这是有问题的,因为我们不知道哪个超平面更好。理论上,它们看起来一样好,但实际上,我们希望选择图\ref{figure25}的超平面。

\begin{figure}[ht]
	\centering
	\includegraphics{figure26}
	\caption{超平面正确错误分类了数据}
	\label{figure26}
\end{figure}

我们如何调整我们的公式来满足这个要求?

好吧,在我们的训练样本$(\mathbf{x},y)$中有一个值我们没有使用:样本的标签$y$。

如果我们往$\beta$值上乘一个$y$,就能改变它的符号,把这个新的值记作$f$:
\begin{gather*}
f = y \times \beta \\
f = y(\mathbf{w}\cdot \mathbf{x} + b)
\end{gather*}

这样如果\textbf{样本被正确分类,则$f$的符号永远会是正号};而如果\textbf{样本被错误分类,则$f$的符号永远会是负号}。

给定数据集$\mathcal{D}$,我们可以计算:
\begin{gather*}
F = \min_{i=1\dots m}f_i \\
F = \min_{i=1\dots m}y_i (\mathbf{w}\cdot\mathbf{x}+b)
\end{gather*}

根据这个公式,当比较两个超平面时,我们仍然会选择其中$F$最大的一个。额外的好处是,在像图\ref{figure25}和图\ref{figure26}这样的特殊情况下,我们将始终选择正确分类的超平面(因为它的值将为正值,而其他超平面的值将为负值)。


在文献中,$f$被称为函数间隔;它的值可以用Python计算,如代码19所示。类似地,这个数字被称为数据集$\mathcal{D}$的函数间隔。

*代码19*

\begin{lstlisting}[language=python]
# Compute the functional margin of an example (x,y) 
# with respect to a hyperplane defined by w and b. 
def example_functional_margin(w, b, x, y): 
    result = y * (np.dot(w, x) + b) 
    return result 
# Compute the functional margin of a hyperplane 
# for examples X with labels y. 
def functional_margin(w, b, X, y): 
    return np.min([example_functional_margin(w, b, x, y[i]) for i, x in enumerate(X)])
\end{lstlisting}

利用这个公式,我们发现在图\ref{figure25}中超平面的函数间隔为+2,而在图\ref{figure26}中超平面的函数间隔为-2。因为图\ref{figure25}中的超平面有更大的函数间隔,所以我们选择第一个。

提示:记住,我们希望选择边界(间隔)最大的超平面


\subsection{标度不变性(Scale invariance)}

看来这次我们找到了比较两个超平面的好方法。然而,函数间隔有一个主要的问题:不具有标度不变性。

给定一个向量$\mathbf{w}_1=(2,1)$和偏移量$b_1=5$,如果把它们乘以10,就得到$\mathbf{w}_2=(20,10)$和$b_2=50$。我们对它们进行了\textbf{重缩放(rescaled)}。

向量$\mathbf{w}_1$和$\mathbf{w}_2$代表同一个超平面,因为它们具有相同的单位向量。超平面是一个正交于向量$\mathbf{w}$的平面,它与向量的长度无关。唯一重要的是它的方向,正如我们在第一章看到的,它是由单位向量给出的。此外,当在图上跟踪超平面时,垂直轴与超平面的交点的坐标将是$0,b/w_1$,因此超平面也不会因为缩放$b$而改变。

正如我们在代码20中所看到的,问题是,当我们用$\mathbf{w}_2$计算函数间隔时,我们得到的是一个比$\mathbf{w}_1$大10倍的数字。这意味着给定任何超平面,只要缩放$\mathbf{w}$和$b$,我们总能找到一个函数间隔更大的超平面。

*代码20*

\begin{lstlisting}[language=python]
x = np.array([1, 1]) 
y = 1 

b_1 = 5 
w_1 = np.array([2, 1]) 

w_2 = w_1 * 10 
b_2 = b_1 * 10 

print(example_functional_margin(w_1, b_1, x, y)) # 8 
print(example_functional_margin(w_2, b_2, x, y)) # 80
\end{lstlisting}

要解决这个问题,我们只需要做一个小小的调整。我们不用向量$\mathbf{w}$,而是用它的单位向量。为此,我们将除以它的模。用同样的方法,我们也对$b$除以$\mathbf{w}$的模使它也保持比例不变。

回想一下函数间隔的公式:$f = y(\mathbf{w}\cdot \mathbf{x} + b)$

稍微对它进行一点修改,得到一个新的数字$\gamma$:
\begin{gather*}
\gamma = y(\frac{\mathbf{w}}{\|w\|}\cdot \mathbf{x} + \frac{b}{\|w\|})
\end{gather*}
和前面一样,给定一个数据集$\mathcal{D}$,我们计算:
\begin{gather*}
M = \min_{i=1\dots m} \gamma_i \\
M = \min_{i=1\dots m} y_i(\frac{\mathbf{w}}{\|w\|}\cdot \mathbf{x}_i + \frac{b}{\|w\|})
\end{gather*}

这样做的好处是,不管我们选择的向量$\mathbf{w}$有多大,结果都是一样的。$\gamma$也被称为样本的\textbf{几何间隔},而$M$则是数据集的几何间隔。代码21是其的一个Python实现。

*代码21*

\begin{lstlisting}[language=python]
# Compute the geometric margin of an example (x,y) 
# with respect to a hyperplane defined by w and b. 
def example_geometric_margin(w, b, x, y): 
    norm = np.linalg.norm(w) 
    result = y * (np.dot(w/norm, x) + b/norm) 
    return result 
    
# Compute the geometric margin of a hyperplane 
# for examples X with labels y. 
def geometric_margin(w, b, X, y): 
    return np.min([example_geometric_margin(w,b, x, y[i]) for i, x in enumerate(X)])
\end{lstlisting}

我们可以验证一下几何间隔是否如我们期望的那样。代码22中,向量$\mathbf{w}_1$和它的重缩放向量$\mathbf{w}_2$都返回了同一个值。

*代码22*

\begin{lstlisting}[language=python]
x = np.array([1,1]) 
y = 1 

b_1 = 5 
w_1 = np.array([2,1])

 w_2 = w_1*10 
 b_2 = b_1*10 

 print(example_geometric_margin(w_1, b_1, x, y)) # 3.577708764 
 print(example_geometric_margin(w_2, b_2, x, y)) # 3.577708764
\end{lstlisting}
它被称为几何间隔,因为我们可以用简单的几何方法验证这个公式。它给出了超平面和$\mathbf{x}$之间的距离。

在图\ref{figure27}中,我们看到这个点$X'$是点$X$到超平面的正交投影。我们希望找到$X$和$X'$之间的距离$d$。

\begin{figure}[ht]
	\centering
	\includegraphics{figure27}
	\caption{点$X$到超平面的几何间隔}
	\label{figure27}
\end{figure}


向量$\mathbf{k}$和向量$\mathbf{w}$有相同的方向,所以他们有一个相同的单位向量$\frac{\mathbf{w}}{\|w\|}$。我们知道$\mathbf{k}$的模是$d$,所以$\mathbf{k}$为:$\mathbf{k}=d\frac{\mathbf{w}}{\|w\|}$。

此外,我们可以看到$\mathbf{x}'=\mathbf{x}-\mathbf{k}$,所以我们把$\mathbf{k}$代入得:
\begin{gather*}
\mathbf{x}'=\mathbf{x}-d\frac{\mathbf{w}}{\|w\|}
\end{gather*}

因为点$mathbf{x}'$是在超平面上,所以$\mathbf{x}'$满足超平面方程,故有:
\begin{gather*}
\mathbf{w} \cdot \mathbf{x}' +b = 0\\
\mathbf{w} \cdot (\mathbf{x} - d\frac{\mathbf{w}}{\|w\|})+b = 0\\
\mathbf{w}\cdot\mathbf{x} - d\frac{\mathbf{w}\cdot\mathbf{w}}{\|w\|}+b=0 \\
\mathbf{w}\cdot\mathbf{x} - d\frac{\|w\|^2}{\|w\|}+b=0 \\
\mathbf{w}\cdot\mathbf{x} - d\|w\| +b=0 \\
d = \frac{\mathbf{w}\cdot\mathbf{x}+b}{\|w\|}\\
d = \frac{\mathbf{w}}{\|w\|}\cdot \mathbf{x} + \frac{b}{\|w\|}
\end{gather*}

最后,和前面一样,我们两边乘上$y$,以确保我们选择了一个正确分类数据的超平面,它给出了我们前面看到的几何间隔公式:
\begin{gather*}
\gamma = y(\frac{\mathbf{w}}{\|w\|}\cdot \mathbf{x} + \frac{b}{\|w\|})
\end{gather*}

\begin{figure}[ht]
    \begin{minipage}{.5\linewidth}
        \centering
	    \includegraphics{figure28}
	    \caption{w=(-0.4,-1)和b=8的超平面}
	    \label{figure28}
    \end{minipage}
    \begin{minipage}{.5\linewidth}
        \centering
	    \includegraphics{figure29}
	    \caption{w=(-0.4,-1)和b=8.5的超平面}
	    \label{figure29}
    \end{minipage}
	
\end{figure}


现在我们已经定义了几何间隔,让我们看看它是如何让我们比较两个超平面的。我们可以看到,与图\ref{figure29}相比,图\ref{figure28}中的超平面更接近蓝色星形示例,而不是红色三角形示例。因此,我们期望它的几何间隔更小。代码23使用代码21中定义的函数来计算每个超平面的几何间隔。从图\ref{figure29}中可以看出,$\mathbf{w}=(-0.4,-1)$和$b=8.5$定义的第二个超平面的几何间隔更大$(0.64 > 0.18)$。在这两者之间,我们会选择图\ref{figure29}这个超平面。

*代码23*

\begin{lstlisting}[language=python]
# Compare two hyperplanes using the geometrical margin. 
positive_x = [[2,7],[8,3],[7,5],[4,4],[4,6],[1,3],[2,5]] 
negative_x = [[8,7],[4,10],[9,7],[7,10],[9,6],[4,8],[10,10]] 

X = np.vstack((positive_x, negative_x)) 
y = np.hstack((np.ones(len(positive_x)), -1*np.ones(len(negative_x)))) 

w = np.array([-0.4, -1]) 
b = 8 

# change the value of b 
print(geometric_margin(w, b, X, y)) # 0.185695338177 
print(geometric_margin(w, 8.5, X, y)) # 0.64993368362
\end{lstlisting}

我们看到,为了计算另一个超平面的几何间隔,我们只需要修改$\mathbf{w}$或$b$的值。我们可以试着用一个小增量来改变它,看看边际是否会变大,但这是一种随机的,它会花很多时间。我们的目标是在所有可能的超平面中为一个数据集找到最优的超平面,但是超平面有无数个。

tip:要找到最优超平面,只要找到$\mathbf{w}$和$b$的值,就可以得到最大的几何间隔。

我们怎么才能找到使几何间隔取得最大的值$\mathbf{w}$呢?幸运的是,数学家已经设计了解决这些问题的工具。为了找到$\mathbf{w}$和$b$,我们需要解决所谓的\textbf{优化问题}。在研究支持向量机的优化问题之前,让我们快速回顾一下什么是优化问题。

\section{什么是优化问题}

\subsection{无约束优化问题}

优化问题的目标是最小化或最大化关于某个变量x的函数(也就是说,找到函数返回的最小或最大值的x值)。例如,求函数$f(x)=x^2$的最小值的问题是这样的:
\begin{gather*}
\underset{x}{\text{minimize}} \quad f(x)
\end{gather*}
或者写作:
\begin{gather*}
\min_x \quad f(x)
\end{gather*}

在这种情况下,我们可以在所有可能的值中自由搜索。我们说这个问题是不受约束的。如图\ref{figure30}所示,函数的最小值在$x=0$处。

\begin{figure}[ht]
    \begin{minipage}{.5\linewidth}
        \centering
	    \includegraphics{figure30}
	    \caption{无约束条件下,最小值是0}
	    \label{figure30}
    \end{minipage}
    \begin{minipage}{.5\linewidth}
        \centering
	    \includegraphics{figure31}
	    \caption{在约束为x-2=0的条件下,最小值为4}
	    \label{figure31}
    \end{minipage}
\end{figure}

\subsection{约束优化问题}

\subsubsection{单等式约束}
有时我们对函数本身的最小值不感兴趣,而是对满足某些约束条件时的最小值感兴趣。在这种情况下,我们通常在约束条件前面加上*subject to*,通常缩写为*s.t.*。例如,如果我们想知道$f$的最小值,但$x$限制为一个特定的值,我们可以这样写:

\begin{gather*}
\underset{x}{\text{minimize}} \quad f(x) \\
subject \ to \quad x=2
\end{gather*}
这个示例如图\ref{figure31}所示。一般来说,等式右边的约束条件是0,所以问题可以被重写:
\begin{gather*}
\underset{x}{\text{minimize}} \quad f(x) \\
subject \ to \quad x-2=0
\end{gather*}

使用这个符号,我们清楚地看到约束是一个\textbf{仿射函数(affine function)},而目标函数是一个\textbf{二次函数(quadratic function)}。因此我们称这个问题为二次优化问题或二次规划(Quadratic Programming,QP)问题。

\subsubsection{可行域(feasible set)}

满足问题约束的变量集称为可行集(或可行域)。在求解优化问题时,将从可行集中选取解。在图\ref{figure31}中,可行集只包含一个值,所以问题很简单。然而,当我们操作具有多个变量的函数时,例如$f(x,y)=(x^2+y^2)$,它允许我们知道我们试图从哪个值中选取最小(或最大值)。

例如:
\begin{gather*}
\underset{x,y}{\text{minimize}} \quad f(x,y) \\
subject \ to \quad x-2=0
\end{gather*}

在这个问题中,可行集是所有点对$(x,y)$的集合,如$(x,y)=(2,y)$

\subsubsection{多等式约束何向量表示法}

我们可以添加任意多的约束条件。这里有一个函数有三个约束条件的问题的例子$f(x,y,z)=x^2+y-z^2$:
\begin{gather*}
\begin{align*}
& \underset{x,y,z}{\text{minimize}} \quad & f(x,y,z) \\
& subject \ to  & x-2=0 \\
& & y+8=0 \\
& & z+3=0
\end{align*}
\end{gather*}

当我们有几个变量时,我们可以切换到向量表示法来提高可读性。对于向量$\mathbf{x}=(x,y,z)^T$,函数变成$f(x)=x_1^2-x_2+x_3^2$,问题就变成:
\begin{gather*}
\begin{align*}
& \underset{\mathbf{x}}{\text{minimize}} \quad & f(\mathbf{x}) \\
& subject \ to  & \mathbf{x}_1-2=0 \\
& & \mathbf{x}_2+8=0 \\
& & \mathbf{x}_3+3=0
\end{align*}
\end{gather*}

当添加约束时,请记住这样做会减少可行集。要接受一个解时,这个解必须满足所有的约束条件。

例如,我们看下面这个问题:
\begin{gather*}
\begin{align*}
& \underset{x}{\text{minimize}} & x^2 \\
& subject \ to  & x-2=0 \\
& & x-8=0 \\
\end{align*}
\end{gather*}

我们可以认为$x=2$和$x=8$是解,但事实并非如此。$x=2$时,约束$x-8=0$不满足;$x=8$时,约束$x-2=0$不满足;这个问题无可行集(或者说无解)。

tip: 如果你给一个问题添加太多的约束条件,它就会变得不可行

如果你通过添加一个约束来改变一个优化问题,你会使优化变得更糟,或者,最好的情况下,新加的约束让原可行集保持不变(Gershwin, 2010)。

\subsubsection{不等式约束}

我们也可以使用不等式作为约束:
\begin{gather*}
\begin{align*}
& \underset{x,y}{\text{minimize}} & x^2+y^2 \\
& subject \ to  & x-2 \geq 0 \\
& &  y \geq 0 \\
\end{align*}
\end{gather*}
我们也可以结合等式约束和不等式约束:
\begin{gather*}
\begin{align*}
& \underset{x,y}{\text{minimize}} & x^2+y^2 \\
& subject \ to  & x-2 = 0 \\
& &  y \geq 0 \\
\end{align*}
\end{gather*}
\subsubsection{如何解决优化问题?}

有几种方法可以解决各种类型的优化问题。然而,介绍它们超出了本书的范围。感兴趣的读者可以查看*OpModels and Application*(El Ghaoui, 2015)和*Convex Optimization*(Boyd \& Vandenberghe, 2004),这两本好书可以开始这个主题,并且可以在线免费获得(详情请参阅参考书目)。相反,我们将再次关注支持向量机,并推导出一个优化问题,使我们能够找到最优超平面。如何解决支持向量机优化问题将在下一章详细解释。

\section{支持向量机的优化问题}

给定一个线性可分数据集$\mathcal{D}=\left\{(\mathbf{x}_i,y_i)\mid \mathbf{x}_i \in \mathbb{R}^n,y_i \in \{+1,-1\}  \right\}_{i=1}^m$以及一个由向量$\mathbf{w}$和偏移量$b$决定的超平面,回忆一下超平面的几何间隔$M$是由以下定义的:
\begin{gather*}
M = \min_{i=1\dots m}\gamma_i
\end{gather*}
其中$\gamma_i=y_i(\frac{\mathbf{w}}{\|w\|}\cdot \mathbf{x}_i + \frac{b}{\|w\|})$是样本$(\mathbf{x}_i,y_i)$的几何间隔。

最优分离超平面是由法向量$\mathbf{w}$和偏移量$b$定义的几何间隔$M$最大的超平面。

为了找到$\mathbf{w}$和$b$,我们需要解决下面的优化问题,约束是每个样本的间隔应该大于或等于$M$:
\begin{gather*}
\underset{\mathbf{w},b}{\text{maximize}} \quad M \\
subject\ to\quad \gamma_i \geq M,i=1,\dots,m
\end{gather*}

几何间隔与函数间隔之间存在一定的关系:
\begin{gather*}
M = \frac{F}{\|w\|}
\end{gather*}

所以我们可以把问题重写为:
\begin{gather*}
\begin{align*}
\underset{\mathbf{w},b}{\text{maximize}} \quad &M \\
subject\ to\quad &\frac{f_i}{\|w\|} \geq \frac{F}{\|w\|},i=1,\dots,m
\end{align*}
\end{gather*}
然后我们可以通过去除不等式两边的模来简化约束:
\begin{gather*}
\begin{align*}
\underset{\mathbf{w},b}{\text{maximize}} \quad &M \\
subject\ to\quad &f_i \geq F,i=1,\dots,m
\end{align*}
\end{gather*}

回想一下,我们试图最大化几何间隔,$\mathbf{w}$和$b$的缩放比例并不重要。我们可以任意缩放$\mathbf{w}$和$b$,其几何间隔并不会改变。因此,我们决定缩放$\mathbf{w}$和$b$直到$F=1$。它不会影响优化问题的结果。(注:这里的M和F都是超平面到数据集的间隔,而不是数据点(或者叫样本)到超平面的间隔。不要记混了)

问题就变成了:
\begin{gather*}
\begin{align*}
\underset{\mathbf{w},b}{\text{maximize}} \quad &M \\
subject\ to\quad &f_i \geq 1,i=1,\dots,m
\end{align*}
\end{gather*}

因为$M=\frac{F}{\|w\|}$,所以有:
\begin{gather*}
\begin{align*}
\underset{\mathbf{w},b}{\text{maximize}} \quad &\frac{F}{\|w\|} \\
subject\ to\quad &f_i \geq 1,i=1,\dots,m
\end{align*}
\end{gather*}
而我们又让$F=1$,故上面等价于:
\begin{gather*}
\begin{align*}
\underset{\mathbf{w},b}{\text{maximize}} \quad &\frac{1}{\|w\|} \\
subject\ to\quad &f_i \geq 1,i=1,\dots,m
\end{align*}
\end{gather*}
这个最大化问题等价于下面的最小化问题:
\begin{gather*}
\begin{align*}
\underset{\mathbf{w},b}{\text{minimize}} \quad &\|w\| \\
subject\ to\quad &y_i(\mathbf{w}\cdot\mathbf{x}_i+b) \geq 1,i=1,\dots,m
\end{align*}
\end{gather*}

> tip:您还可以在[本页](http://www.svm-tutorial.com/2015/06/svm-understanding-math-part-3/)上阅读这个优化问题的另一个推导,在这里我使用几何间隔,而不是函数间隔和几何间隔。

这个最小化问题给出了与下面相同的结果:

\begin{gather*}
\begin{align*}
\underset{\mathbf{w},b}{\text{minimize}} \quad &\frac{1}{2}\|w\|^2 \\
subject\ to\quad &y_i(\mathbf{w}\cdot\mathbf{x}_i+b) \geq 1,i=1,\dots,m
\end{align*}
\end{gather*}
为了方便以后使用QP求解器来解决问题,添加了这个$\frac{1}{2}$因子,并且对范数进行平方具有去掉平方根的优点。

最后,你会看到写在大多数文献里的优化问题:
\begin{gather*}
\begin{align*}
\underset{\mathbf{w},b}{\text{minimize}} \quad &\frac{1}{2}\|w\|^2 \\
subject\ to\quad &y_i(\mathbf{w}\cdot\mathbf{x}_i+b) \geq 1,i=1,\dots,m
\end{align*}
\end{gather*}

为什么我们要这么费劲地重写这个问题呢?因为原来的优化问题很难解决。相反,我们现在有一个凸二次优化问题(convex quadratic optimization problem),虽然不明显,但更容易解决。

\section{总结}

首先,我们假设某些超平面比其他超平面更好:它们在处理新的数据时会表现得更好。在所有可能的超平面中,我们决定将“最佳”超平面称为最优超平面。为了找到最优超平面,我们寻找了一种比较两个超平面的方法,最后我们得到了一个允许我们这样做的数。我们发现这个数字也有几何意义,叫做几何间隔。

然后我们说,最优超平面是具有最大几何间隔的超平面,我们可以通过最大化间隔来找到它。为了让事情变得简单,我们注意到,我们可以最小化$\mathbf{w}$的模,即超平面的法向量,我们可以确定它是最优超平面的$\mathbf{w}$(如果你记得,$\mathbf{w}$在计算几何间隔的公式中使用过)。
\include{4SolvingTheOP}
\chapter{软间隔支持向量机}

\section{处理噪声数据}

硬间隔支持向量机最大的问题是它要求数据是线性可分的。现实生活中的数据常常是线性不可分的。即使数据是线性可分的,在将其输入模型之前也会发生很多事情。也许有人在示例中输入了错误的值,或者可能传感器的探测返回了一个异常的值。在存在异常值(离该类别的大部分数据点很远)的情况下,有两种情况:异常值可以比该类的大多数样本更接近其他类别的样本,从而导致间隔的减少,或者它打破其他类别的线性可分性。 让我们硬间隔SVM是如何处理这两种情况的。

\subsection{异常值减小间隔}

当数据线性可分时,硬间隔分类器在存在异常值时不会像我们希望的那样。

现在让我们考虑添加异常数据点(5,7)的数据集,如图\ref{figure33}所示。

\begin{figure}[ht]
	\centering
	\includegraphics{figure33}
	\caption{添加异常点(5,7)后,数据依然是线性可分的}
	\label{figure33}
\end{figure}

在本例中,我们可以看到间隔非常窄,似乎异常值是这一变化的主要原因。直观地,我们可以看到这个超平面可能不是分离数据的最佳超平面,而且它的泛化能力可能很差。

\subsection{异常值让数据线性不可分}

更糟糕的是,当异常值打破线性可分性时,如图\ref{figure34}中的点(7,8),分类器无法找到超平面。我们被一个单一的数据点卡住了

\begin{figure}[ht]
	\centering
	\includegraphics{figure34}
	\caption{异常点(7,8)让数据线性不可分}
	\label{figure34}
\end{figure}


\section{软间隔代解决方案}

\subsection{松弛变量}

1995年,Vapnik和Cortes引入了原始支持向量机的改进版本,允许分类器犯一些错误。现在的目标不是零分类错误,而是犯尽可能少的错误。

为此,他们通过添加一个变量$\zeta$(zeta)来修改优化问题的约束条件。约束:
\begin{gather*}
y_i (\mathbf{w}\cdot \mathbf{x}_i + b) \geq 1
\end{gather*}
变成了:
\begin{gather*}
y_i (\mathbf{w}\cdot \mathbf{x}_i + b) \geq 1-\zeta_i
\end{gather*}

因此,在最小化目标函数时,即使样本不满足原始约束(即离超平面太近,或不在超平面的正确一侧),也有可能满足该约束。代码29说明了这一点。

\emph{代码29}

\begin{lstlisting}[language=python]
import numpy as np 

w = np.array([0.4, 1]) 
b = -10 

x = np.array([6, 8]) 
y = -1 

def constraint(w, b, x, y): 
    return y * (np.dot(w, x) + b)

def hard_constraint_is_satisfied(w, b, x, y): 
    return constraint(w, b, x, y) >= 1 
    
def soft_constraint_is_satisfied(w, b, x, y, zeta): 
    return constraint(w, b, x, y) >= 1 - zeta 
    
# While the constraint is not satisfied for the example (6,8). 
print(hard_constraint_is_satisfied(w, b, x, y)) # False 

# We can use zeta = 2 and satisfy the soft constraint. 
print(soft_constraint_is_satisfied(w, b, x, y, zeta=2)) # True
\end{lstlisting}

问题是,我们可以为每个样本选择一个很大的$\zeta$值,这样所有的约束条件都会得到满足。

\emph{代码30}

\begin{lstlisting}[language=python]
# We can pick a huge zeta for every point 
# to always satisfy the soft constraint. 
print(soft_constraint_is_satisfied(w, b, x, y, zeta=10)) # True 
print(soft_constraint_is_satisfied(w, b, x, y, zeta=1000)) # True
\end{lstlisting}

为了避免这种情况,我们需要修改目标函数以对$\zeta$值大的进行惩罚:
\begin{gather*}
\begin{align*}
\underset{\mathbf{w},b,\zeta}{\text{minimize}} \quad &\frac{1}{2}\|w\|^2 + \sum_{i=1}^m \zeta_i \\
subject\ to\quad &y_i(\mathbf{w}\cdot\mathbf{x}_i+b) \geq 1 - \zeta_i,i=1,\dots,m
\end{align*}
\end{gather*}
我们把所有个体的$\zeta$总和加到目标函数中。添加这样的惩罚称为\textbf{正则化}。因此,解决方案将是在具有最小误差的情况下最大化j间隔超平面。

还有一个小问题。有了这个公式,我们可以很容易地利用的负值$\zeta_i$来最小化函数。我们添加约束$\zeta_i \geq 0$来防止这种情况。此外,我们希望在软间隔方面保持一定的控制。也许有时我们想要使用硬间隔—毕竟,这就是我们添加参数$C$的原因,它将帮助我们确定$\zeta$有多重要(稍后详细介绍)。

这就引出了\textbf{软间隔公式}:

\begin{gather*}
\begin{align*}
\underset{\mathbf{w},b,\zeta}{\text{minimize}} \quad &\frac{1}{2}\|w\|^2 + \sum_{i=1}^m \zeta_i \\
subject\ to\quad &y_i(\mathbf{w}\cdot\mathbf{x}_i+b) \geq 1 - \zeta_i \\
& \zeta_i \geq 0,i=1,\dots,m
\end{align*}
\end{gather*}

如(Vapnik V. N., 1998)所示,使用与线性可分情况相同的方法,我们发现我们需要在一个稍微不同的约束下,最大化相同的对偶问题:
\begin{gather*}
\begin{align*}
\underset{\alpha}{\text{maximize}} \quad & \sum_{i=1}^m \alpha_i - \frac{1}{2}\sum_{i=1}^m\sum_{j=1}^m \alpha_i \alpha_j y_i y_j \mathbf{x}_i \cdot \mathbf{x}_j  \\
subject\ to \quad & 0 \leq \alpha_i \leq C,\text{for any }i=1,\dots,m \\
& \sum_{i=1}^m \alpha_i y_i = 0
\end{align*}
\end{gather*}

这里的约束从$\alpha_i \geq 0$变成了$0 \leq \alpha_i \leq C$。这个约束通常被称为框约束(box constraint),因为向量$\alpha$被限制在边长为$C$正交的框内。注意,正交是平面上象限的模拟$n$维欧几里德空间(Cristianini \& Shawe-Taylor, 2000)。我们将在关于SMO算法的章节中的图\ref{figure50}中的可视化框约束

因为我们最小化松弛向量$\zeta$的1-范式,优化问题也称为\textbf{1-范式软间隔}。



\section{理解参数C的作用}

这个参数$C$让你可以控制SVM如何处理错误。现在让我们看看改变它的值将得到何种不同的超平面。

图\ref{figure35}显示了我们在本书中使用的线性可分数据集。在左边,我们可以看到把$C$设置为$+\infty$得到了与硬间隔分类器相同的结果。然而,如果我们选择一个更小的值$C$,就像我们在中间图做的那样,我们可以看到超平面比左侧的超平面更接近一些点。这些点违反了硬间隔约束。令$C=0.01$加剧了这种行为,如右侧图所示。

\begin{figure}[ht]
	\centering
	\includegraphics{figure35}
	\caption{线性可分的数据集中分别设置$C=+\infty,C=1,C=0.01$}
	\label{figure35}
\end{figure}

如果我们选择一个非常接近零的值$C$会发生什么?那么基本上就没有约束了,我们得到的超平面不能分类任何样本。

当数据是线性可分的时候,坚持用大$C$是最好的选择。但如果我们有一些嘈杂的异常值呢?在这种情况下,正如我们在图\ref{figure36}中所看到的,使用$C=+\infty$得到了一个非常窄的间隔。然而,当我们使用$C=1$时,我们得到的超平面与没有异常值下的硬间隔超平面非常接近。只有异常值违反了约束。设置$C=0.01$时则有一个非异常值违反了约束。这个$C$的值似乎给了我们的软间隔分类器太多的自由。

\begin{figure}[ht]
	\centering
	\includegraphics{figure36}
	\caption{线性可分的数据集中添加一个异常值后,分别设置$C=+\infty,C=1,C=0.01$}
	\label{figure36}
\end{figure}

最终,在异常值使数据线性不可分的情况下,我们不能使用$C=+\infty$,因为没有满足所有硬边距约束的解。相反,我们测试了$C$的几个值,并看到使用$C=3$得到了最好的超平面。事实上,我们在$C \geq 3$时得到的超平面都是一样的。这是因为无论我们如何惩罚它,都必须违反异常值的约束才能分离数据。和前面一样,当我们使用小$C$时,会违反更多的约束。

\begin{figure}[ht]
	\centering
	\includegraphics{figure36}
	\caption{线性不可分的数据集中,分别设置$C=3,C=1,C=0.01$}
	\label{figure36}
\end{figure}

经验法则:

* 较小的$C$会带来更大的间隔,但会存在一些错误的分类。
* 较大的$C$更接近硬间隔分类器,在这种情况下基本不能违反约束条件
* 找到噪声数据不会对解产生太大影响的$C$很关键。



\section{怎么找到最好的参数$C$}

没有任何$C$可以解决所有的问题。推荐的选择方法是使用\href{http://scikit-learn.org/stable/modules/cross_validation.html}{交叉验证}的\href{http://scikit-learn.org/stable/modules/grid_search.html}{网格搜索}(Hsu, Chang, \& Lin, A Practical Guide to Support Vector Classification)来选择$C$。要明白$C$的值非常特定于您正在使用的数据,所以如果有一天你发现$C = 0.001$不适合你的数据集,你仍然应该尝试将这个值用在另一个数据集中,因为相同的$C$在不同的数据集中完全不同。

\section{其他软间隔公式}

\subsection{2-范式软间隔(2-Norm soft margin)}

这个问题还有另一种形式,叫做2-范式(或L2正则化)软间隔,它最小化的是$\frac{1}{2} \|w\|^2 + C \sum\limits_{i=1}^m \zeta_i^2$。这个公式引出了一个没有框约束的对偶问题。关于2-范式软间隔的更多信息,请参阅(Cristianini \& shaw - taylor, 2000)。

\subsection{nu-SVM}

由于$C$的大小受特征空间的影响,所以提出了$\nu SVM$。其思想是使用一个值在0到1之间变化的参数$\nu$,而不是参数$C$。

> Note: “$\nu$为问题提供了一个更透明的参数化,它不依赖于特征空间的缩放,而只依赖于数据中的噪声水平。”(Cristianini \& Shawe-Taylor, 2000)

其所要解决的优化问题为:
\begin{gather*}
\begin{align*}
\underset{\alpha}{\text{maximize}} \quad &  - \frac{1}{2}\sum_{i=1}^m\sum_{j=1}^m \alpha_i \alpha_j y_i y_j \mathbf{x}_i \cdot \mathbf{x}_j  \\
subject\ to \quad & 0 \leq \alpha_i \leq \frac{1}{m},\text{for any }i=1,\dots,m \\
& \sum_{i=1}^m \alpha_i y_i = 0 \\
& \sum_{i=1}^m \alpha_i \geq \nu,i=1,\dots,m
\end{align*}
\end{gather*}
\section{总结}

相对于硬间隔分类器,软间隔支持向量机是一个很好的改进。即使有噪声数据打破线性可分性,它依然允许我们正确地分类数据。然而,这种增加的灵活性的代价是我们现在有了一个超参数$C$,我们需要为它找到一个值。我们看到改变$C$的值是如何影响间隔的,并允许分类器为了获得更大的间隔而做一些错位分类。这再次提醒我们,我们的目标是找到一种可以很好地处理新数据的假设函数。在训练数据上出现一些错误并不是一件坏事。
\chapter{核}

\section{特征转换(Feature transformatioins)}

\subsection{我们能对线性不可分数据进行分类吗?}

假设您有一些线性不可分的数据(如图\ref{figure38}中的数据),您希望使用svm对其进行分类。我们已经知道这是不可能的,因为数据不是线性可分的。然而,最后这个假设是不正确的。这里需要注意的是数据在\textbf{二维空间}中不是线性可分的。

\begin{figure}[ht]
	\centering
	\includegraphics{figure38}
	\caption{直线不能分离这些数据}
	\label{figure38}
\end{figure}


即使原始数据是二维的,也不能阻止您在将其输入SVM之前对其进行转换。例如,一种可能的变换是,将每个二维向量$(x_1,x_2)$转换为三维向量。

例如,我们可以通过定义的函数$\phi :\mathbb{R}^2->\mathbb{R}^3$来实现多项式映射:
\begin{gather*}
\phi (x_1,x_2) = (x_1^2,\sqrt{2}x_1 x_2,x_2^2)
\end{gather*}

代码31显示了用Python实现的这个转换。

\emph{代码31}

\begin{lstlisting}[language=python]
# Transform a two-dimensional vector x into a three-dimensional vector. 
def transform(x): 
    return [x[0]**2, np.sqrt(2)*x[0]*x[1], x[1]**2]
\end{lstlisting}

如果您转换图\ref{figure38}的整个数据集并绘制结果,就会得到图\ref{figure39},它并没有显示出多大的改进。然而,经过一段时间的处理,我们可以看到数据实际上在三维空间中是可分离的(图\ref{figure40}和图\ref{figure41})!

\begin{figure}[ht]
	\centering
	\includegraphics{figure39}
	\caption{在三维空间中看起来不能分割}
	\label{figure39}
\end{figure}

\begin{figure}[ht]
	\centering
	\includegraphics{figure40}
	\caption{数据实际可以被平面分割}
	\label{figure40}
\end{figure}

\begin{figure}[ht]
	\centering
	\includegraphics{figure41}
	\caption{另一个视角可以看到 数据在平面的两侧}
	\label{figure41}
\end{figure}

下面是一个基本的方法,我们可以使用它来分类这个数据集:

1. 用代码31把二维向量转换为三维向量
2. 对3维数据集进行训练
3. 对于我们希望预测的每个新样本,在将其传递给预测方法之前,使用`transform`方法对其进行转换

当然,您不必强制将数据转换为三维;它可以是5维,10维,或者100维

\subsection{我们如何知道要使用哪个转换}

选择应用哪种转换很大程度上取决于您的数据集。 能够转换数据以使您希望使用的机器学习算法发挥最佳性能可能是机器学习领域成功的关键因素之一。 不幸的是,没有完美的配方,它会通过反复试验获得经验。 在使用任何算法之前,请务必检查是否有一些通用规则来转换文档中详述的数据。 有关如何准备数据的更多信息,您可以阅读 scikit-learn 网站上的\href{http://scikit-learn.org/stable/data_transforms.html}{数据集转换}部分。


\section{什么是核}

在上一节中,我们看到了一个用于线性不可分数据集的方法。它的一个主要缺点是我们必须转换每个样本。如果我们有数百万或数十亿个样本,而转换方法很复杂,那就会花费大量的时间。这时,核就来救场了。

如果你还记得,当我们在对偶拉格朗日函数中寻找KKT乘子时,我们不需要一个训练样本$\mathbf{x}$的值;我们只需要两个训练样本之间的点积$\mathbf{x}_i \cdot \mathbf{x}_j$:
\begin{gather*}
W(\alpha) = \sum_{i=1}^m \alpha_i - \frac{1}{2}\sum_{i=1}^m\sum_{j=1}^m \alpha_i \alpha_j y_i y_j \mathbf{x}_i \cdot \mathbf{x}_j
\end{gather*}

在代码32中,我们应用诀窍的第一步。想象一下,当数据被用来学习时,我们唯一关心的是点积返回的值,在这个例子中是8100。

\emph{代码32}

\begin{lstlisting}[language=python]
x1 = [3,6] 
x2 = [10,10] 

x1_3d = transform(x1) 
x2_3d = transform(x2)

print(np.dot(x1_3d,x2_3d)) # 8100

\end{lstlisting}
问题是:\textbf{有没有一种方法可以在不变换向量的情况下,计算出这个值?}

答案是:是的,用核!

让我们看看代码33中的函数:

\emph{代码33}

\begin{lstlisting}[language=python]
def polynomial_kernel(a, b): 
    return a[0]**2 * b[0]**2 + 2*a[0]*b[0]*a[1]*b[1] + a[1]**2 * b[1]**2
\end{lstlisting}

在前面的两个示例中使用这个函数将返回相同的结果(代码34)。

\emph{代码34}

\begin{lstlisting}[language=python]
x1 = [3,6] 
x2 = [10,10] 

# We do not transform the data. 

print(polynomial_kernel(x1, x2)) # 8100
\end{lstlisting}

仔细想想,这是非常不可思议的。

向量$\mathbf{x}_1$和$\mathbf{x}_2$都是二维的,核函数计算它们的点积,就好像它们被转换成了三维的向量一样,它不需要做变换,也不需要计算它们的点积!

总结一下:核是一个函数,它返回在另一个空间中执行的点积的结果。更正式的说法是:

\textbf{定义}:给定一个映射函数:$\phi: \mathcal{X} \rightarrow \mathcal{V}$,我们把由$K(\mathbf{x},\mathbf{x}')=\langle \phi(\mathbf{x}),\phi(\mathbf{x}')\rangle_\mathcal{V}$定义的函数$K:\mathcal{X} \rightarrow \mathbb{R}$,称为\textbf{核函数}。其中$\langle\cdot,\cdot\rangle_\mathcal{V}$表示$\mathcal{V}$的内积

\subsection{核技巧}

现在我们知道了什么是核,我们将看到什么是核技巧。

如果我们定义了一个核:$K(\mathbf{x}_i,\mathbf{x}_j) = \mathbf{x}_i \cdot \mathbf{x}_j$,我们可以重写软间隔的对偶问题:
\begin{gather*}
\begin{align*}
\underset{\alpha}{\text{maximize}} \quad & \sum_{i=1}^m \alpha_i - \frac{1}{2}\sum_{i=1}^m\sum_{j=1}^m \alpha_i \alpha_j y_i y_j K(\mathbf{x}_i \cdot \mathbf{x}_j)  \\
subject\ to \quad & 0 \leq \alpha_i \leq C,\text{for any }i=1,\dots,m \\
& \sum_{i=1}^m \alpha_i y_i = 0
\end{align*}
\end{gather*}

就是这样。我们对对偶问题做了一个简单的改变——我们称之为核技巧。

> Tip:应用内核技巧仅仅意味着用一个核函数替换两个样本的点积

这个变化看起来非常简单,但请记住,要从最初的优化问题推导出对偶公式需要做大量的工作。我们现在有能力改变核函数来分类线性不可分的数据

当然,我们还需要改变假设函数来使用核函数:

\begin{gather*}
h(\mathbf{x}_i) = sign(\sum_{j=1}^S \alpha_j y_j K(\mathbf{x}_j \cdot \mathbf{x}_i)+b)
\end{gather*}

记住这个公式中的$S$是支持向量的集合。通过这个公式,我们可以更好地理解为什么支持向量机也被称为稀疏内核机。这是因为它们只需要在支持向量上计算核函数,而不是像其他核方法那样在所有的向量上计算核函数(Bishop, 2006)。

\section{核的类型}

\subsection{线性核}
这是最简单的核。它是这样的:
\begin{gather*}
K(\mathbf{x},\mathbf{x}') = \mathbf{x} \cdot \mathbf{x}'
\end{gather*}
其中$\mathbf{x},\mathbf{x}'$是两个向量。

在实践中,您应该知道线性核可以很好地用于\href{http://www.svm-tutorial.com/2014/10/svm-linear-kernel-good-text-classification/}{文本分类}

\subsection{多项式核}
我们在之前介绍核的时候已经看到了多项式核,但这一次我们将考虑更通用的版本:
\begin{gather*}
K(\mathbf{x},\mathbf{x}') = (\mathbf{x} \cdot \mathbf{x}' + c)^d
\end{gather*}
它有两个参数,一个$c$表示常数项,一个$d$表示核的次数。这个很容易用python实现,如代码35所示。

\emph{代码35}

\begin{lstlisting}[language=python]
def polynomial_kernel(a, b, degree, constant=0): 
    result = sum([a[i] * b[i] for i in range(len(a))]) + constant 
    return pow(result, degree)
\end{lstlisting}

在代码36中,我们看到当我们使用次数2时,它返回的结果与代码33的相同。使用该核训练SVM的结果如图\ref{figure42}所示。

\emph{代码36}

\begin{lstlisting}[language=python]
x1 = [3,6] 
x2 = [10,10] 

# We do not transform the data. 
print(polynomial_kernel(x1, x2, degree=2)) # 8100
\end{lstlisting}

\begin{figure}[ht]
	\centering
	\includegraphics{figure42}
	\caption{使用多项式核的SVM能够分离数据(degree=2)}
	\label{figure42}
\end{figure}

次数为1且没有常数的多项式核就是线性核(图\ref{figure43})。当增加多项式核的次数时,决策边界将变得更加复杂,并且有受个别数据示例影响的趋势,如图\ref{figure44}所示。使用高次多项式是危险的,因为高次多项式通常可以在测试集上获得更好的性能,但它会导致所谓的\textbf{过拟合}:模型太接近数据,不能很好地泛化。


\begin{figure}[ht]
	
    \begin{minipage}{.4\linewidth}
        \centering
	    \includegraphics{figure43}
	    \caption{degree=1的多项式核}
	    \label{figure43}
    \end{minipage}
    \begin{minipage}{.4\linewidth}
    	\centering
	    \includegraphics{figure44}
	    \caption{degree=6的多项式核}
	    \label{figure44}
    \end{minipage}
\end{figure}


> Note:使用高次多项式核常常会导致过拟合

\subsection{RBF核(高斯核)}

有时候多项式核还不够复杂。当您有一个如图\ref{figure45}所示的复杂数据集时,这种类型的核有其局限性。

\begin{figure}[ht]
	\centering
	\includegraphics{figure45}
	\caption{更复杂的数据集}
	\label{figure45}
\end{figure}

正如我们在图\ref{figure46}中所看到的,决策边界在分类数据方面非常糟糕

\begin{figure}[ht]
	\centering
	\includegraphics{figure46}
	\caption{多项式核不能分离这个数据(degree=3,C=100)}
	\label{figure46}
\end{figure}

这种情况需要我们使用另一种更复杂的核:高斯核。它也被称为RBF核,其中RBF代表径向基函数。径向基函数是其值仅取决于到原点或到某一点的距离的函数

其公式为:
\begin{gather*}
K(\mathbf{x},\mathbf{x}') = \exp(-\gamma\|\mathbf{x}-\mathbf{x}'\|^2)
\end{gather*}

你会经常在其他地方读到它将向量映射到无限维空间中。这是什么意思?

在我们前面看到的多项式核的例子中,核返回在$\mathbb{R}^3$中执行的点积的结果。RBF内核返回在$\mathbb{R}^\infty$中执行的点积的结果。

我不会在这里详述,但如果你愿意,你可以阅读这个\href{http://pages.cs.wisc.edu/~matthewb/pages/notes/pdf/svms/RBFKernel.pdf}{证明},以更好地理解我们是如何得出这个结论的。

\begin{figure}[ht]
	\centering
	\includegraphics{figure47}
	\caption{gamma=0.1时,RBF核正确分类了数据}
	\label{figure47}
\end{figure}

\href{https://www.youtube.com/watch?v=3liCbRZPrZA}{这个视频}对于理解RBF内核如何分离数据特别有用。

\begin{figure}[ht]
	\centering
	\includegraphics{figure48}
	\caption{gamma=1e-5时的RBF核}
	\label{figure48}
\end{figure}

\begin{figure}[ht]
	\centering
	\includegraphics{figure49}
	\caption{gamma=2时的RBF核}
	\label{figure49}
\end{figure}


当gamma值太小时,如图\ref{figure48}所示,模型表现为线性SVM。当gamma值太大时,模型受每个支持向量的影响太大,如图\ref{figure49}所示。有关gamma的更多信息,您可以阅读\href{http://scikit-learn.org/stable/auto_examples/svm/plot_rbf_parameters.html}{scikit-learn文档页面}。

\subsection{其他核}

对核的研究一直很丰富,现在有很多可用的核。其中一些是特定于域的,比如\href{https://en.wikipedia.org/wiki/String_kernel}{字符串核},可以在处理文本时使用。如果您想了解更多的核,这篇来自César Souza的\href{http://crsouza.com/2010/03/17/kernel-functions-for-machine-learning-applications/}{文章}描述了25种内核

\section{该怎么去使用核}

推荐的方法是先尝试RBF内核,因为它通常表现得很好。但是,如果有足够的时间,最好尝试其他类型的内核。核是两个向量之间相似度的度量,因此,这就是当前问题的领域知识可能产生最大影响的地方。构建自定义核也是可能的,但这需要您对核背后的理论有良好的数学理解。你可以在(Cristianini \& shaw - taylor, 2000)中找到关于这个主题的更多信息。

\section{总结}

核技巧是使支持向量机强大的一个关键组件。它允许我们将支持向量机应用于各种各样的问题。在本章中,我们看到了线性核的局限性,以及多项式核如何分类线性不可分的数据。最终,我们看到了最常用、最强大的核之一:RBF核。不要忘记有许多核,并尝试寻找为解决您要解决的问题而创建的核。使用正确的核和正确的数据集是支持向量机成败的一个关键因素。
\chapter{SMO算法}

我们看到了如何使用凸优化包来解决SVM优化问题。然而,在实践中,我们将使用一种专门为快速解决这个问题而创建的算法:SMO(序列最小优化)算法。大多数机器学习库使用SMO算法或其他变体。

SMO算法将解决以下优化问题:

\begin{gather*}
\begin{align*}
\underset{\alpha}{\text{minimize}} \quad & \frac{1}{2}\sum_{i=1}^m\sum_{j=1}^m \alpha_i \alpha_j y_i y_j K(\mathbf{x}_i \cdot \mathbf{x}_j) -\sum_{i=1}^m \alpha_i \\
subject\ to \quad & 0 \leq \alpha_i \leq C,\text{for any }i=1,\dots,m \\
& \sum_{i=1}^m \alpha_i y_i = 0
\end{align*}
\end{gather*}

它是我们在第5章中看到的软间隔公式的核化版本。我们试图最小化的目标函数可以用Python编写(代码37):

\emph{代码37}

\begin{lstlisting}[language=python]
def kernel(x1, x2): 
    return np.dot(x1, x2.T) 

def objective_function_to_minimize(X, y, a, kernel): 
    m, n = np.shape(X) 
    return 1 / 2 * np.sum([a[i] * a[j] * y[i] * y[j]* kernel(X[i, :], X[j, :]) for j in range(m) for i in range(m)])\ - np.sum([a[i] for i in range(m)])
\end{lstlisting}

这和我们使用CVXOPT解决的问题是一样的。为什么我们需要另一种方法?因为我们希望能够在大数据集上使用支持向量机,而使用凸优化包通常涉及到矩阵操作,随着矩阵大小的增加,这些操作需要花费大量时间,或者由于内存限制而变得不可能。创建SMO算法的目标是比其他方法更快。

\section{SMO背后的理念}
当我们尝试解决SVM优化问题时,只要我们满足约束条件,我们可以自由地改变的$\alpha$值。我们的目标是修改$\alpha$,使目标函数最终返回最小的可能值。在这种情况下,给定一个拉格朗日乘子向量$\alpha=(\alpha_1,\alpha_2,\dots,\alpha_m)$,我们可以改变任意值$\alpha_i$,直到我们达到目标。

SMO背后的思想非常简单:我们将解决一个更简单的问题。也就是说,给定一个向量$\alpha=(\alpha_1,\alpha_2,\dots,\alpha_m)$,我们只允许自己改变$\alpha$的两个值,例如,$\alpha_3$和$\alpha_7$。我们将改变它们,直到目标函数达到给定的最小值。然后,我们将选择另外两个alpha并更改它们,直到函数返回其最小值,以此类推。如果我们继续这样做,最终会得到原问题的目标函数的最小值。

> 注:如果仅修改一个$\alpha_i$的值,就会违反约束条件$\sum\limits_{i=1}^m \alpha_i y_i = 0$,所以要两个一起修改。

SMO解决了一系列简单的优化问题

\section{How did we get to SMO}

这种解决几个更简单的优化问题的想法并不新鲜。1982年,Vapnik提出了一种称为“分块(chunking)”的方法,将原始问题分解成一系列更小的问题(Vapnik V., 1982)。让事情发生变化的是,1997年,Osuna等人证明了只要我们加入至少一个违反KKT条件的例子,求解一系列子问题将保证收敛(Osuna, Freund, \& Girosi, 1997)。

利用这个结果,一年后,也就是1998年,Platt提出了SMO算法。

\section{SMO为什么快}

SMO方法的最大优点是我们不需要QP求解器来求解两个拉格朗日乘子的问题——它可以解析求解。因此,它不需要存储一个巨大的矩阵,从而导致机器内存出现问题。此外,SMO使用了几种启发式算法来加快计算速度。

\section{SMO算法 }

SMO算法由三部分组成:

* 根据启发式方法选择第一个拉格朗日乘子 
* 根据启发式方法选择第二个拉格朗日乘子 
* 根据选择的两个乘子,用代码求优化问题的解析解

> Tip: 该算法的Python实现可在附录B: SMO算法中获得。本节中的所有代码清单都摘自本附录,并不是单独运行的。

\subsection{解析解}

在算法的开始前,我们把向量$\alpha(\alpha_1,\alpha_2,\dots,\alpha_m)$初始化为0,即$\alpha_i=0,for\ all\ i=1,\dots,m$。其思想是从这个向量中选择两个元素,我们将其命名为$\alpha_1$和$\alpha_2$,并更改它们的值,以使约束条件仍然满足。

第一个约束条件$0 \leq \alpha_i \leq C ,for\ all\ i=1,\dots,m$意味着$0 \leq \alpha_1 \leq C,0 \leq \alpha_2 \leq C$。这就是为什么我们选择图\ref{figure50}中蓝色框内的值(其中$C=5$)

第二个约束是一个线性约束$\sum\limits_{i=1}^m \alpha_i y_i = 0$。它的取值范围位于红色对角线上,并且之前被选中的乘子$\alpha_1$和$\alpha_2$应该有不同的标签($y_1 \neq y_2$)。

\begin{figure}[ht]
	\centering
	\includegraphics{figure50}
	\caption{可行集是蓝色区域的对角线}
	\label{figure50}
\end{figure}

一般来说,为了避免打破线性约束,我们必须改变乘子,像这样:
\begin{gather*}
\alpha_1 y_1 + \alpha_2 y_2 = constant = \alpha_1^{old} y_1 + \alpha_2^{old} y_2
\end{gather*}

我们不会深入分析问题是如何解决的细节,因为它在(Cristianini \& shaw - taylor, 2000)和(Platt J. C., 1998)做得很好。

记住,有一个公式来计算新的$\alpha_2$:
\begin{gather*}
\alpha_2^{new} = \alpha_2 + \frac{y_2(E_1-E_2)}{K(\mathbf{x}_1,\mathbf{x}_1)+K(\mathbf{x}_2,\mathbf{x}_2)-2K(\mathbf{x}_1,\mathbf{x}_2)}
\end{gather*}

其中$E_i = f(\mathbf{x}_i)-y_i$是假设函数的输出和样本标签之间的差值。$K$是核函数。我们还计算了适用于$\alpha_2^{new}$的边界;它不能小于下界,也不能大于上界,否则会违反约束。在这种情况下$\alpha_2^{new}$被修剪掉了。

> 注:这里的修剪的意思大致如下: 
\begin{lstlisting}[language=python]
ifa>C:
    a = C
elif a<0:
    a = 0
\end{lstlisting}

一旦我们有了这个新值,我们就用它来计算$\alpha_1$的新值:
\begin{gather*}
\alpha_1^{new} = \alpha_1^{old}+y_1 y_2 (\alpha_2^{old}-\alpha_2^{new})
\end{gather*}

\subsection{理解第一个启发式选择}

第一个启发式背后的想法非常简单:每次 SMO 检查一个样本时,它都会检查是否违反了 KKT 条件。 回想一下,至少有一个样本违反了KKT条件(注:如果全都满足KKT条件,那就是最优解了,不用再求了)。 如果满足条件,则尝试另一个样本。 因此,如果有数百万个示例,而其中只有少数违反了 KKT 条件,它将花费大量时间检查无用的示例。 为了避免这种情况,该算法将时间集中在拉格朗日乘数不等于$0$或不等于$C$的样本上,因为它们最有可能违反条件(代码 38)。

\emph{代码38}

\begin{lstlisting}[language=python]
def get_non_bound_indexes(self): 
    return np.where(np.logical_and(self.alphas > 0, self.alphas < self.C))[0] 
    
# First heuristic: loop over examples where alpha is not 0 and not C 
# they are the most likely to violate the KKT conditions 
# (the non-bound subset). 
def first_heuristic(self): 
    num_changed = 0 
    non_bound_idx = self.get_non_bound_indexes() 
    
    for i in non_bound_idx: 
        num_changed += self.examine_example(i) 
    return num_changed
\end{lstlisting}

由于解析解问题涉及两个拉格朗日乘数,有可能一个约束乘数(其值在$0$和$C$之间)已经违反了KKT条件。这就是主程序在所有样本和非边界(non-bound)子集之间交替的原因(代码39)。注意,当不再有进展时,算法结束

\subsection{理解第二个启发式选择}

第二种启发式选择的目标是选择变化最大的拉格朗日乘子。

怎么更新$\alpha_2$呢,我们使用之前的公式:
\begin{gather*}
\alpha_2^{new} = \alpha_2 + \frac{y_2(E_1-E_2)}{K(\mathbf{x}_1,\mathbf{x}_1)+K(\mathbf{x}_2,\mathbf{x}_2)-2K(\mathbf{x}_1,\mathbf{x}_2)}
\end{gather*}

记住,在这个例子中,我们已经选择了$\alpha_1$值。我们的目标是选出那些将会使$\alpha_2$最大改变的。这个公式可以改写为:

\begin{gather*}
\alpha_2^{new} = \alpha_2 + step \\
step = \frac{y_2(E_1-E_2)}{K(\mathbf{x}_1,\mathbf{x}_1)+K(\mathbf{x}_2,\mathbf{x}_2)-2K(\mathbf{x}_1,\mathbf{x}_2)}
\end{gather*}

所以,为了从几个$\alpha_2$中选出最好的$\alpha_2$,我们需要计算每个$\alpha_i$的步长并选择步长最大的那个。这里的问题是每一步我们需要调用核函数$K$三次,这代价很高。Platt没有这么做,而是提出了以下近似:
\begin{gather*}
step \approx |E_1 -E_2|
\end{gather*}
因此,当$E_1$为正时,那么选择最小的$E_i$作为$E_2$;如果$E_1$为负,选择最大$E_i$作为$E_2$。

在代码40的方法\colorbox{lightgray}{second\_heuristic}中可以看到这种选择。

\emph{代码40}

\begin{lstlisting}[language=python]
def second_heuristic(self, non_bound_indices): 
    i1 = -1 
    if len(non_bound_indices) > 1: 
        max = 0 

    for j in non_bound_indices: 
        E1 = self.errors[j] - self.y[j] 
        step = abs(E1 - self.E2) # approximation 
        if step > max: 
            max = step 
            i1 = j 
        return i1 
        
def examine_example(self, i2): 
    self.y2 = self.y[i2] 
    self.a2 = self.alphas[i2] 
    self.X2 = self.X[i2] 
    self.E2 = self.get_error(i2) 
    
    r2 = self.E2 * self.y2 
    
    if not((r2 < -self.tol and self.a2 < self.C) or (r2 > self.tol and self.a2 > 0)): 
        # The KKT conditions are met, SMO looks at another example. 
        return 0 
    
    # Second heuristic A: choose the Lagrange multiplier that 
    # maximizes the absolute error. 
    non_bound_idx = list(self.get_non_bound_indexes()) 
    i1 = self.second_heuristic(non_bound_idx) 
    
    if i1 >= 0 and self.take_step(i1, i2): 
        return 1 
        
    # Second heuristic B: Look for examples making positive 
    # progress by looping over all non-zero and non-C alpha, 
    # starting at a random point. 
    if len(non_bound_idx) > 0: 
        rand_i = randrange(len(non_bound_idx)) 
        for i1 in non_bound_idx[rand_i:] + non_bound_idx[:rand_i]: 
            if self.take_step(i1, i2): 
                return 1 
                
    # Second heuristic C: Look for examples making positive progress 
    # by looping over all possible examples, starting at a random
    # point. 
    rand_i = randrange(self.m) 
    all_indices = list(range(self.m)) 
    for i1 in all_indices[rand_i:] + all_indices[:rand_i]: 
        if self.take_step(i1, i2): 
            return 1 
    
    # Extremely degenerate circumstances, SMO skips the first example. 
    return 0
\end{lstlisting}

\section{总结}

理解SMO算法可能很困难,因为这里有很多代码是出于性能原因,或者是为了处理特定的退化情况。然而,其核心的算法不难,而且比凸优化求解速度更快。随着时间的推移,人们已经发现了新的启发式算法来改进该算法,而且像LIBSVM这样的流行库使用了类似于smo的算法。注意,即使这是解决SVM问题的标准方法,也存在其他方法,如梯度下降和随机梯度下降(stochastic gradient descent,SGD),这用于在线学习和处理巨大的数据集。

了解SMO算法如何工作将帮助您确定它是否是您想要解决的问题的最佳方法。我强烈建议你自己试试。在斯坦福CS229课程中,你可以找到一个\href{http://cs229.stanford.edu/materials/smo.pdf)的描述,这是一个很好的开始。然后,在序列最小化(Platt J. C., 1998}{简化版算法},你可以阅读算法的完整描述。附录B中提供的Python代码是用本文的伪代码编写的,并在注释中指出代码的哪些部分对应于本文中的哪些方程。
\chapter{多分类SVM}

支持向量机能够生成二分类器。然而,我们经常会遇到数据集有两个以上的类别。例如,原始的葡萄酒数据集实际上包含来自三个不同生产商的数据。有几种方法允许支持向量机进行多分类。在本章中,我们将回顾一些最流行的多类方法,并解释它们的来源。

对于本章中的所有代码示例,我们将使用代码41生成的数据集,并显示在图\ref{figure51}中。

\emph{代码41}

\begin{lstlisting}[language=python]
import numpy as np 

def load_X(): 
    return np.array([[1, 6], [1, 7], [2, 5], [2, 8], [4, 2], [4, 3], [5, 1], [5, 2], [5, 3], [6, 1], [6, 2], [9, 4], [9, 7], [10, 5], [10, 6], [11, 6], [5, 9], [5, 10], [5, 11], [6, 9], [6, 10], [7, 10], [8, 11]]) 

def load_y(): 
    return np.array([1, 1, 1, 1, 2, 2, 2, 2, 2, 2, 2, 3, 3, 3, 3, 3, 4, 4, 4, 4, 4, 4, 4])

\end{lstlisting}

\begin{figure}[ht]
	\centering
	\includegraphics{figure51}
	\caption{四分类问题}
	\label{figure51}
\end{figure}



\section{解决多个二分类问题}
\subsection{一对多(One-against-all)}

这可能是最简单的方法,也被称为“一个对其余的”(one-versus-the-rest)。

为了对$K$个类进行分类,我们构造了$K$个不同的二分类器。对于给定的类,正样本是该类中的所有数据点,负样本是除该类外的所有数据点(代码42)。

\emph{代码42}

\begin{lstlisting}[language=python]
import numpy as np 
from sklearn import svm 

# Create a simple dataset 

X = load_X() 
y = load_y() 

# Transform the 4 classes problem 
# in 4 binary classes problems. 
y_1 = np.where(y == 1, 1, -1) 
y_2 = np.where(y == 2, 1, -1) 
y_3 = np.where(y == 3, 1, -1) 
y_4 = np.where(y == 4, 1, -1)

\end{lstlisting}

我们针对每个问题训练一个二分类器(代码43)。结果,我们得到每个分类器一个决策边界(如图\ref{figure52}所示)。

\emph{代码43}

\begin{lstlisting}[language=python]
# Train one binary classifier on each problem. 
y_list = [y_1, y_2, y_3, y_4] 

classifiers = [] 
for y_i in y_list: 
    clf = svm.SVC(kernel='linear', C=1000) 
    clf.fit(X, y_i) 
    classifiers.append(clf)

\end{lstlisting}

\begin{figure}[ht]
	\centering
	\includegraphics{figure52}
	\caption{一对多方法为每个类别创建了一个分裂器}
	\label{figure52}
\end{figure}

为了做出新的预测,我们将新的样本运用在所有分类器上,如果某个分类器返回的是正值,则表明该样本是属于返回正值分类器的类别(代码44)。然而,因为一个标签被同时分配给多个类别或没有分配到一个类别(Bishop, 2006),可能会产生不一致的结果。图\ref{figure53}说明了这个问题;一对多的分类器不能为蓝色区域中的样本预测一个类别,因为存在两个分类器在蓝色区域都返回的是正值。这将导致样本同时拥有两个类别。同样的问题也出现在中间,因为每个分类器都在中间的区域中返回负值。因此,不能将任何类分配给该区域中的样本。

\emph{代码44}

\begin{lstlisting}[language=python]
def predict_class(X, classifiers): 
    predictions = np.zeros((X.shape[0], len(classifiers))) 
    for idx, clf in enumerate(classifiers): 
        predictions[:, idx] = clf.predict(X) 
        
    # returns the class number if only one classifier predicted it 
    # returns zero otherwise. 
    return np.where((predictions == 1).sum(1) == 1, (predictions == 1).argmax(axis=1) + 1, 0)

\end{lstlisting}

\begin{figure}[ht]
	\centering
	\includegraphics{figure53}
	\caption{一对多导致了模糊的结果}
	\label{figure53}
\end{figure}

作为一种替代解决方案,Vladimir Vapnik建议使用决策函数值最大的分类器的类(Vapnik V. N, 1998)。代码45演示了这一点。注意,我们使用了\colorbox{lightgray}{decision\_function}而不是调用分类器的\colorbox{lightgray}{predict}方法。该方法返回一个实值,如果样本位于分类器的正确一侧,则为正,如果位于分类器的另一侧,则为负。值得注意的是,当所有分类器不一致时,通过取最大值而不是绝对值的最大值,这种方法将选择最接近样本超平面的类。例如,图\ref{figure54}中的数据点(6,4)将被分配为蓝色的星形类。

\emph{代码45}

\begin{lstlisting}[language=python]
def predict_class(X, classifiers): 
    predictions = np.zeros((X.shape[0], len(classifiers))) 
    for idx, clf in enumerate(classifiers): 
        predictions[:, idx] = clf.decision_function(X) 
    
    # return the argmax of the decision function as suggested by Vapnik.
    return np.argmax(predictions, axis=1) + 1
\end{lstlisting}

\begin{figure}[ht]
	\centering
	\includegraphics{figure54}
	\caption{应用简单的启发式方法避免了模糊决策问题}
	\label{figure54}
\end{figure}

应用这种启发式方法,我们得到的分类结果没有歧义,如图\ref{figure54}所示。这种方法的主要缺陷是不同的分类器是在不同的任务上训练的,所以不能保证\colorbox{lightgray}{decision\_function}返回的数量具有相同的尺度(Bishop, 2006)。如果一个\colorbox{lightgray}{decision}函数返回的结果比其他\colorbox{lightgray}{decision}函数的结果大10倍,那么在某些例子中,它会分到错位的类别中。

“一对多”方法的另一个问题是,训练集是不平衡的(Bishop, 2006)。对于100个类,假设每个类别有10个样本,那么每个分类器都将用10个正样本和990个负样本训练。因此,负样本对决策边界的影响很大。

尽管如此,“一对多”仍然是多类分类的流行方法,因为它易于实现和理解。

> Note: 在实践中,一对其余(one-versus-the-rest)的办法是使用最广泛的,尽管它的公式特别,在实际中也有局限性。(Bishop, 2006)

当使用sklearn时,LinearSVC默认自动使用"一对多"策略。您还可以通过将\colorbox{lightgray}{multi\_class}参数设置为\colorbox{lightgray}{ovr}(one-vs-the-rest)来显式地指定它,如代码46所示。

\emph{代码46}

\begin{lstlisting}[language=python]
from sklearn.svm import LinearSVC 
import numpy as np 

X = load_X() 
y = load_y() 

clf = LinearSVC(C=1000, random_state=88, multi_class='ovr') 
clf.fit(X,y) 

# Make predictions on two examples. 
X_to_predict = np.array([[5,5],[2,5]]) 
print(clf.predict(X_to_predict)) # prints [2 1]
\end{lstlisting}

\subsection{一对一(one-against-one)}

在这种方法中,我们不是试图将一个类别与所有其他类别区分开来,而是试图将一个类别与另一个类别区分开来。因此,我们对每一对类别训练一个分类器,得到K个类别的K(K-1)/2个分类器。每个分类器都是在数据的一个子集上训练的,并产生自己的决策边界(图\ref{figure55})。

\begin{figure}[ht]
	\centering
	\includegraphics{figure55}
	\caption{一对一 为每一对类别构建了分类器}
	\label{figure55}
\end{figure}

预测是使用一种简单的\textbf{投票策略(voting strategy)}进行的。我们希望预测的每个样本都被传递给每个分类器,预测的类被记录下来。然后,拥有最多票数的类被分配给样本(代码47)。

\emph{代码47}

\begin{lstlisting}[language=python]
from itertools import combinations 
from scipy.stats import mode 
from sklearn import svm 
import numpy as np 

# Predict the class having the max number of votes. 
def predict_class(X, classifiers, class_pairs): 
    predictions = np.zeros((X.shape[0], len(classifiers))) 
    for idx, clf in enumerate(classifiers): 
        class_pair = class_pairs[idx] 
        prediction = clf.predict(X) 
        predictions[:, idx] = np.where(prediction == 1, class_pair[0], class_pair[1]) 
    return mode(predictions, axis=1)[0].ravel().astype(int) 

X = load_X() 
y = load_y() 

# Create datasets. 
training_data = [] 
class_pairs = list(combinations(set(y), 2)) 
for class_pair in class_pairs: 
    class_mask = np.where((y == class_pair[0]) | (y == class_pair[1])) 
    y_i = np.where(y[class_mask] == class_pair[0], 1, -1) 
    training_data.append((X[class_mask], y_i)) 

# Train one classifier per class. 
classifiers = [] 
for data in training_data: 
    clf = svm.SVC(kernel='linear', C=1000) 
    clf.fit(data[0], data[1]) 
    classifiers.append(clf) 
    
# Make predictions on two examples. 
X_to_predict = np.array([[5,5],[2,5]]) 
print(predict_class(X_to_predict, classifiers, class_pairs)) #prints [2 1]
\end{lstlisting}


在这种方法中,我们仍然面临着模糊的分类问题。如果两个类有相同的投票数量,就建议选择索引较小的那一个(注:就是数组中靠前的那一个)可能是可行的(但可能不是最好的)策略(Hsu \& Lin, a Comparison of Methods for Multi-class Support Vector Machines, 2002)。

\begin{figure}[ht]
	\centering
	\includegraphics{figure56}
	\caption{预测是使用投票方案进行的}
	\label{figure56}
\end{figure}


图\ref{figure56}中显示了一对一策略生成的决策区域与一对多策略生成的决策区域是不同的(图\ref{figure54})。在图\ref{figure57}中,有趣的是,对于一对一分类器生成的区域,区域只有在经过超平面(用黑线表示)后才会改变其颜色,而对于一对多所有的情况则不是这样。

\begin{figure}[ht]
	\centering
	\includegraphics{figure57}
	\caption{一对多(左)和一对一(右)的比较}
	\label{figure57}
\end{figure}

一对一的方法是sklearn中使用的多类分类的默认方法。与代码47不同,使用代码48的代码将获得与代码47中完全相同的结果。

\emph{代码48}

\begin{lstlisting}[language=python]

from sklearn import svm 
import numpy as np 

X = load_X() 
y = load_y() 

# Train a multi-class classifier. 
clf = svm.SVC(kernel='linear', C=1000) 
clf.fit(X,y) 

# Make predictions on two examples. 
X_to_predict = np.array([[5,5],[2,5]]) 
print(clf.predict(X_to_predict)) # prints [2 1]
\end{lstlisting}

针对一对多的方法的一个主要缺点是分类器会倾向于过拟合。此外,分类器的个数随类别的数量呈超线性增长,因此该方法在处理大型问题时速度较慢(Platt, Cristianini, \& shaw - taylor, 2000)。


\subsection{有向无环图支持向量机(Directed Acyclic Graph SVM,DAGSVM)}

它是由John Platt等人在2000年提出的,作为一对一的改进(Platt, Cristianini, \& shaw - taylor, 2000)。

> Note:John C. Platt发明了SMO算法和Platt Scaling,并提出了DAGSVM。这是对svm世界的巨大贡献!

DAGSVM背后的思想是使用与一对一相同的训练,但通过使用有向无环图(DAG)选择使用哪个分类器来加快测试速度。

如果我们有四个类别,分别是:A, B, C, D,和六个分别训练在一对类上的分类器:(A,B);(A,C);(A,D);(B,C);(B,D)和(C,D)。我们对样本S使用第一个分类器(A,D),其预测是A而不是D,则用第二个分类器(A,C)继续预测,得到的结果不是C。这意味着分类器(B, D)、(B, C)或(C, D)可以忽略,因为我们已经确定样本S不属于类别C或类别D.最后用分类器(A,B),如果它预测结果是B,我们就本B类分配给样本S。图中的红色路径说明了这个例子。图的每个节点是一对类的分类器。

\begin{figure}[ht]
	\centering
	\includegraphics{figure58}
	\caption{用于沿有向无环图进行预测的路径说明}
	\label{figure58}
\end{figure}

对于四个类,我们使用三个分类器来进行预测,而不是一对一中的六个分类器。一般来说,对于有K个类的问题,将评估K-1个决策节点。

将代码47中的\colorbox{lightgray}{predict\_class}函数替换为代码49中的\colorbox{lightgray}{predict\_class}函数可以得到相同的结果,但可以使用更少的分类器。

在代码49中,我们用一个列表实现DAGSVM方法。我们从可能的类别列表开始,在每次预测之后,我们删除不合格的类别。最后,剩下的类是应该分配给样本的类。

注意,这里的代码49是为了演示的目的,不应该在您的生产代码中使用,因为当数据集(X)很大时,它并不快。

\emph{代码49}

\begin{lstlisting}[language=python]
def predict_class(X, classifiers, distinct_classes, class_pairs): 
    results = [] 
    for x_row in X: 
        
        class_list = list(distinct_classes) 
        
        # After each prediction, delete the rejected class 
        # until there is only one class. 
        while len(class_list) > 1: 
            # We start with the pair of the first and 
            # last element in the list. 
            class_pair = (class_list[0], class_list[-1]) 
            classifier_index = class_pairs.index(class_pair) 
            y_pred = classifiers[classifier_index].predict(x_row) 
            
            if y_pred == 1: 
                class_to_delete = class_pair[1] 
            else: class_to_delete = class_pair[0] 
            
            class_list.remove(class_to_delete) 
        
        results.append(class_list[0]) 
    return np.array(results)

\end{lstlisting}

> Note: "DAGSVM的评估速度比Max Wins快1.6到2.3倍。"(Platt, Cristianini, \& Shawe-Taylor, 2000).

\section{解决单一优化问题}

另一种方法是尝试解决单一优化问题,而不是尝试解决几个二次优化问题。多年来,已经有几个人提出了这种方法。

\subsection{Vapnik, Weston, and Watkins}

该方法是对支持向量机优化问题的一种推广,可直接解决多分类问题。它是由Vapnik (Vapnik V. N, 1998)和Weston \& Watkins (Weston \& Watkins, 1999)独立发现的。把每个类的约束条件都添加到优化问题中。因此,问题的规模与类别的数量成正比,可能会训练的非常慢。

\subsection{Crammer and Singer}

Crammer和Singer (C\&S)提出了一种可选的方法来处理多分类支持向量机。像Weston和Watkins一样,他们解决了一个单一的优化问题,但使用了较少的松弛变量(Crammer \& Singer, 2001)。这可以减少内存和训练时间。然而,在他们的比较研究中,Hsu和Lin发现C\&S方法在使用较大的正则化参数值$C$时特别慢(Hsu和Lin, a Comparison of Methods for Multi-class Support Vector Machines, 2002)。

在sklearn中,当使用LinearSVC时,你可以选择使用C\&S算法(代码50)。在图\ref{figure59}中,我们可以看到C\&S预测不同于一对多的方法和一对一的方法。

\emph{代码50}

\begin{lstlisting}[language=python]
from sklearn import svm 
import numpy as np 

X = load_X() 
y = load_y() 

clf = svm.LinearSVC(C=1000, multi_class='crammer_singer')
clf.fit(X,y) 

# Make predictions on two examples. 
X_to_predict = np.array([[5,5],[2,5]]) 
print(clf.predict(X_to_predict)) # prints [4 1]

\end{lstlisting}

\begin{figure}[ht]
	\centering
	\includegraphics{figure59}
	\caption{Crammer \& Singer算法}
	\label{figure59}
\end{figure}

\section{该使用哪种方法}

有这么多可用的选项,选择一个适合您问题多分类方法可能会很困难。

Hsu和Lin写了一篇有趣的论文,比较了支持向量机的不同多类方法(Hsu和Lin, A Comparison of Methods for multi-class Support Vector Machines, 2002)。他们得出的结论是“一对一和DAG方法比其他方法更适合实际应用。”一对一的方法有一个额外的优势,那就是sklearn中已经有了这种方法,所以它应该是您的默认选择。

一定要记住,LinearSVC在默认情况下使用的是一对多的方法,也许使用Crammer \& Singer算法会更好地帮助你实现目标。在这个问题上,Dogan等人发现,尽管它比其他算法快得多,但针对所有算法的产量假设在统计上的准确性明显较差(Dogan, Glasmachers, \& Igel, 2011)。表1提供了本章中介绍的方法的概述,以帮助您做出选择。

\begin{tabular}{cccccc}
    方法名 & 一对一 & 一对多 & Weston\&Watkins & 有向无环图SVM & C\&S \\

    第一次使用时间 & 1995 & 1996 & 1999 & 2000 & 2001 \\
方法 & 多个二分类 & 多个二分类 & 求解单一优化问题 & 多个二分类 & 求解单一优化问题 \\
训练方式 & 每个类别单独一个分类器 & 每对类别单独一个分类器 & 分解法 & 和一对一相同 & 分解法 \\
分类器数量(K是类别数量) & K & $\frac{K(K-1)}{2}$ & 1 & $\frac{K(K-1)}{2}$ & 1 \\
测试方法 & 选择决策函数值最大的类 & “Max-Wins”投票策略 & 使用分类器 & 使用DAG中的K-1个分类器 & 使用分类器 \\
scikitlearn & LinearSVC & SVC & 无 & 无 & LinearSVC \\
缺点 & 类别样本数量不平衡 & K过大时,训练时间过长 & 训练时间过长 & 常用库中没有 & 训练时间过长 \\
\end{tabular}
\section{总结}

由于多年来的许多改进,现在有几种利用支持向量机进行多类分类的方法。每种方法都有优点和缺点,大多数情况下,您最终将使用正在使用的库中可用的一种方法。但是,如果有必要,您需要知道哪种方法更有助于解决您的特定问题。

关于多分类支持向量机的研究还没有结束。最近关于这个问题的论文集中于分布式训练。例如,Han \& Berg提出了一种名为“分布式共识多类支持向量机(Distributed Consensus Multiclass SVM)”的新算法,该算法使用的是共识优化和Crammer \& Singer公式的修改版本(Han \& Berg, 2012)。


\chapter{结语}

最后,我想引用Stuart russell和Peter Norvig的一句话:

“You could say that SVMs are successful because of one key insight, one neat trick.”

(Russell \& Norvig, 2010)

关键的观点是,有些样本比其他样本更重要。它们是最接近决策边界的,我们称它们为\textbf{支持向量}。结果表明,最优超平面比其他超平面具有更好的泛化性,并且可以只使用支持向量来构造。我们需要解决一个凸优化问题来找到这个超平面。

这个巧妙的技巧(neat trick)是\textbf{核技巧}。它允许我们对线性不可分的数据使用支持向量机,如果没有它,支持向量机将非常有限。我们发现这个技巧虽然一开始很难掌握,但实际上非常简单,可以在其他学习算法中重复使用。

就是这样。如果您已经从头到尾地阅读了这本书,那么您现在应该理解支持向量机是如何工作的了。另一个有趣的问题是为什么它们会起作用?这是一个叫做计算学习理论的领域的主题(支持向量机实际上来自于统计学习理论)。如果你想了解更多,你可以学习\href{http://work.caltech.edu/telecourse.html}{这门}优秀的课程或阅读Learning from Data(Abu-Mostafa, 2012),它有一个很好的课程介绍。

您应该知道支持向量机不仅仅用于分类。单类支持向量机可以用于异常检测,支持向量回归(Support Vector Regression)可以用于回归。为简洁起见,本书没有将它们包括在内,但它们同样是有趣的主题。现在您已经理解了基本的支持向量机,您应该为研究这些推导做了更好的准备。

支持向量机不会解决你所有的问题,但我确实希望它们现在能成为你的机器学习工具箱中的一个工具——一个你能理解并喜欢使用的工具。
\chapter{附录A:数据集}

\section{线性可分数据集}

下面的代码用于加载本书大多数章节中使用的简单线性可分数据集。你可以在\href{https://bitbucket.org/syncfusiontech/svm-succinctly}{这里}中找到本书中使用的其他数据集的源代码。

\begin{figure}[ht]
	\centering
	\includegraphics{figure60}
	\caption{训练集}
	\label{figure60}
\end{figure}

\begin{figure}[ht]
	\centering
	\includegraphics{figure61}
	\caption{测试集}
	\label{figure61}
\end{figure}

当如代码51所示导入模块时,它会加载代码52所示的方法。

方法\colorbox{lightgray}{get\_training\_examples}返回图\ref{figure60}所示的数据,而方法\colorbox{lightgray}{get\_test\_examples}返回图\ref{figure61}所示的数据。

\emph{代码51}

\begin{lstlisting}[language=python]
from succinctly.datasets import *
\end{lstlisting}

\emph{代码52}

\begin{lstlisting}[language=python]
import numpy as np 
def get_training_examples(): 
    X1 = np.array([[8, 7], [4, 10], [9, 7], [7, 10], [9, 6], [4, 8], [10, 10]])
    y1 = np.ones(len(X1)) 
    X2 = np.array([[2, 7], [8, 3], [7, 5], [4, 4], [4, 6], [1, 3], [2, 5]]) 
    y2 = np.ones(len(X2)) * -1 
    return X1, y1, X2, y2 
    
def get_test_examples(): 
    X1 = np.array([[2, 9], [1, 10], [1, 11], [3, 9], [11, 5], [10, 6], [10, 11], [7, 8], [8, 8], [4, 11], [9, 9], [7, 7], [11, 7], [5, 8], [6, 10]]) 
    X2 = np.array([[11, 2], [11, 3], [1, 7], [5, 5], [6, 4], [9, 4],[2, 6], [9, 3], [7, 4], [7, 2], [4, 5], [3, 6], [1, 6], [2, 3], [1, 1], [4, 2], [4, 3]])
    y1 = np.ones(len(X1)) 
    y2 = np.ones(len(X2)) * -1 
    return X1, y1, X2, y2

\end{lstlisting}

代码53展示了这段代码的典型用法。它使用代码54中的\colorbox{lightgray}{get\_dataset}方法,该方法和\colorbox{lightgray}{datasets}包一起加载。

\emph{代码53}

\begin{lstlisting}[language=python]
from succinctly.datasets import get_dataset, linearly_separable as ls 

# Get the training examples of the linearly separable dataset. 
X, y = get_dataset(ls.get_training_examples)
\end{lstlisting}

\emph{代码54}

\begin{lstlisting}[language=python]
import numpy as np 
def get_dataset(get_examples): 
    X1, y1, X2, y2 = get_examples() 
    X, y = get_dataset_for(X1, y1, X2, y2) 
    return X, y 
    
def get_dataset_for(X1, y1, X2, y2): 
    X = np.vstack((X1, X2)) 
    y = np.hstack((y1, y2)) 
    return X, y 
    
def get_generated_dataset(get_examples, n): 
    X1, y1, X2, y2 = get_examples(n) 
    X, y = get_dataset_for(X1, y1, X2, y2) 
    return X, y
\end{lstlisting}

\include{11AppendixB}

\end{document}